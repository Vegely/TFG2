\begin{titlepage}
\begin{center}

%forma de introducir imágenes. el \\[0.5 cm] de final de línea introduce un salto de ese tamaño.
%width=1\textwidth indica el tamaño de la imágen (valores entre 0-1). 
 \includegraphics[width=1\textwidth]{figuras/cabecera.png}  \\[0.5 cm]

\LARGE UNIVERSIDAD POLITÉCNICA DE MADRID \\ [1 cm]

\LARGE ESCUELA TÉCNICA SUPERIOR DE INGENIERÍA Y DISEÑO INDUSTRIAL \\ [1 cm]

\LARGE Grado en Ingeniería Electrónica Industrial y Automática\\ [1 cm]

\LARGE \textbf{TRABAJO FIN DE GRADO}\\[1 cm]

\Huge \textsc{Estudio comparativo de diferentes algoritmos de encriptación para comunicaciones industriales}\\[1 cm]

\LARGE Bogurad Barañski Barañska \\[1.5 cm]

%flushleft alinea a la izquierda el texto

\begin{multicols}{2} 
\begin{flushleft} \large	
\emph{Cotutor:} Basil Mohammed Al-Hadithi\\
\emph{Departamento:} ingeniería eléctrica, electrónica, automática y física aplicada.
\end{flushleft}

\begin{flushright} \large	
\emph{Tutor:}  Roberto Gonzalez Herranz\\
\emph{Departamento:} ingeniería eléctrica, electrónica, automática y física aplicada.
\end{flushright}

\end{multicols} 

%rellena de blanco el resto de la página para escribir abajo del todo
\vfill

% Bottom of the page
{\large Madrid, Septiembre, 2025}

%SE ponen al final firmas.tex
%\end{center}
%\end{titlepage}


\cleardoublepage 