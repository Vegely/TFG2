\chapter{Desarrollo}
\section{Implementación comunicación serie}
\subsection{Parámetros generales y formato mensajes}
\subsection{Implementación en el ordenador}
\subsection{Implementación en el microprocesador}
\section{Interfaz para los algoritmos de cifrado asimétrico} 
Para hacer más sencillo el uso de los distintos algoritmos se crea un \esquote{wrapper} o envoltorio a cada uno. De esta manera, las implementaciones descargadas de la página del NIST \cite{nistPQCround3} \cite{nistPQCround4}. La estructura general de todos los ficheros \texttt{AlgorythWrapper.h} es:


\section{Implementación algoritmos en el PC}
\subsection{Compilación en librerías}
\subsection{Diagrama de uso}
\subsection{Diagrama funcional}
\subsection{Diagrama de clases}
\section{Implementación de algoritmos en el microcontrolador}
\subsection{Diagrama de uso}
\subsection{Diagrama funcional}

\section{Implementación del intercambio de claves. Creación del secreto compartido}
Hablar de los modelos de comunicaciones a implementar .... (msg teams)
\subsection{Modelo 1}
\subsection{Modelo 2}
\subsection{Modelo 3}
\section{Tests de rendimiento realizados}