\chapter{Definiciones básicas}
\label{chap:def}
\chapter{Ejemplo \gls{rlwe}}
\label{chap:ej:rlwe}
Sea el espacio de trabajo en $R_{17}=\mathbb{Z}_{17}[X]/\left(X^2+1\right)$ con un mensaje $z\in \{0,1\}^2$ y la distribución del error $e\in \{-1,0,1\}$.
\newline

En el primer paso se generan $a, s$ y $e$:
\begin{equation}
	\begin{array}{l}
		a[X]=3+4X\\
		s[X]=1+0X\\
		e[X]=-1+1X
	\end{array}
\end{equation}

Una vez inicializados los parámetros, se procede al cálculo de \(b\). La reducción módulo $X^2+1$ equivale a sustituir $X^2$ por $-1$ cada vez que aparezca.
\begin{equation}
	b[X]=a[X]\cdot s[X]+e[X]=2+5X
\end{equation}

Con la clave pública calculada \(a||b\) se puede cifrar un mensaje \(z\), pero antes se generan los valores de \(z, r, e_1\) y \(e_2\):
\begin{equation}
	\begin{array}{l}
		z[X]=1+0X\\
		r[X]=1+1X\\
		e_1[X]=0+1X\\
		e_2[X]=-1+0X
	\end{array}
\end{equation}

Con estos valores se calculan los textos cifrados:
\begin{equation}
	\begin{array}{l}
		u[X]=a[X]\cdot r[X]+e_1[X]=16+8X\\
		v[X]=b[X]\cdot r[X]+e_2[X]+\left\lfloor \dfrac{q}{2} \right\rceil \cdot z[X] =5+7X
	\end{array}
\end{equation}

Por último, se comprueba que el mensaje se descifra correctamente. 
\begin{equation}
	z'[X]=v[X]-u[X]\cdot s[X]=6+16X \rightarrow \left\{ \begin{array}{l}
		z'_0: \ \begin{array}{l}
			\text{d}_0(0) = 6 \\
			\text{d}_0(9) = 3
		\end{array} \\
		z'_1: \ \begin{array}{l}
			\text{d}_1(0) = 1 \\
			\text{d}_1(9) = 8
		\end{array}
	\end{array} \right.
\end{equation}

Con estas distancia se obtiene que \(z=(1,0)\). No obstante, aunque este descifrado se comprueba que se cumple que el error no supera la magnitud límite $q/4=4.25$.
\begin{equation}
	\varepsilon [X]=r[X]\cdot e[X]- s[X]\cdot e_1[X] +e_2[X]=14+16X
\end{equation}

Para cumplirse la distancia a \(0\) debe ser menor a $q/4$ para cada coeficiente:
\begin{equation}
	\begin{array}{l}
		\text{d}_0(0) = 3 \\
		\text{d}_1(0) = 1
	\end{array} 
\end{equation}