\chapter{Estado del arte}
Los artículos que de momento tengo que son relevantes mencionar, lo haré después de los fundamentos:
\cite{NIST_IR_8413_2022} \cite{NIST_IR_8545_2025} \cite{011318_1_5.0179566} \cite{s11042-024-20535-x} \cite{A_COMPARATIVE_REVIEW_OF_DATA_ENCRYPTION_METHODS_IN_THE_USA_AND_EUROPE} \cite{An_Overview_and_Analysis_of_Hybrid_Encryption_The_Combination_of_Symmetric_Encryption_and_Asymmetric_Encryption} \cite{Comparative_Analysis_of_Energy_Costs_of_Asymmetric_vs_Symmetric_Encryption-Based_Security_Applications} \cite{First-Order-Masked-Kyber-on-ARM-Cortex-M4} \cite{2022-414} \cite{An_overview_of_Quantum_Cryptography_and} \cite{2502.12252v1} \cite{Quantum_Resistance_Saber-Based_Group_Key_Exchange_Protocol_for_IoT} \cite{NISTFIPS203}
\section{Introducción}
a
\subsection{Tipos de cifrado}
a
\subsection{Necesidad del cifrado asimétrico}
a
\section{Sistemas de intercambio de claves}
a
\section{Cambio en el paradigma de cifrado asimétrico. Algoritmo de Shore}
a
\subsection{Algoritmo de Shore}
Generico: \cite{9508027v2} ECC: \cite{0301141v2} RSA: \cite{RESERCHFINAL}
\section{Evolución de los algoritmos postcuánticos}
a fundamentos y distintas alternativas a codebased \cite{SurveyCOdeBAsed}
\section{Cifrado asimétrico postcuántico en sistemas embebidos}
a
\section{ALgoritmos postcuanticos}

Kyber has a very efficient key-generation procedure (see also Section 2) and is therefore particularly well
suited for applications that use frequent key generations to achieve forward secrecy. \cite{kyber-spec-2021}