\chapter{Fundamentos generales}
En este capítulo se desarrollan las bases matemáticas de los distintos algoritmos a implementar.
\section{Introducción}
\section{Algoritmos de Hashing }
Para elaborar esta seccion se sigue la seccion de recomendaciones del nist \cite{FIPS202}
\section{Métodos clásicos de cifrado asimétrico}
\subsection{Algoritmo Rivest-Shamir-Adleman (\gls{rsa})}
\subsection{Criptografía en Curvas Elípticas (\gls{ecc})}
\subsection{Algoritmo de Shore}
Generico: \cite{9508027v2} ECC: \cite{0301141v2} RSA: \cite{RESERCHFINAL}
\section{Fundamentos de seguridad de los algoritmos}
Recomendaciones NIST para intercambio seguro de claves \cite{NIST_SP_800_227_ipd_2025}
\subsection{Generación de números aleatorios criptográficamente seguros}
\subsection{Indistinguibilidad bajo ataque de texto cifrado adaptable \gls{cca2}}
\subsection{Transformadas Fujisaki-Okamoto \gls{tfo}}
\cite{Fujisaki1999}
\newpage
\section{Funcionamiento básico de los algoritmos postcuánticos}
En esta sección se describe el funcionamiento de los algoritmos postcuánticos analizados en este trabajo. Dado que no se desarrollaron implementaciones propias, sino que se utilizó el código proporcionado por el NIST en la tercera \cite{nistPQCround3} y cuarta \cite{nistPQCround4} ronda del proceso de estandarización, resulta apropiado presentar su funcionamiento aquí en lugar de en la sección de desarrollo.
\subsection{CRYSTALS-Kyber }
Se utilizará la misma notación empleada en el artículo \cite{kyber-spec-2021}. El conjunto de los enteros sin signo de 8 bits se denota por \(\mathcal{B} = \{0, \dots, 255\}\). Para representar vectores de tamaño \(k\), se utiliza la notación \(\mathcal{B}^k\), mientras que para vectores de tamaño arbitrario se emplea \(\mathcal{B}^*\).
\newline

Para trabajar con estos vectores, se utiliza el símbolo \(||\) para denotar la concatenación de dos cadenas, y la notación \(+k\) para indicar el desplazamiento de \(k\) bytes desde el inicio de una cadena. Por ejemplo, si se tiene una cadena \(a\) de longitud \(l\) y se concatena con una cadena \(b\), se obtiene:
\begin{equation}
	c = a || b
\end{equation}

Entonces:
\begin{equation}
	b = c + l
\end{equation}

Para denotar vectores, se utiliza la notación \(v[i]\), donde \(v\) es un vector columna e \(i\) indica la posición del elemento (empezando desde 0, si no se indica lo contrario). Para las matrices, se emplea la notación \(A[i][j]\), donde \(i\) representa la fila y \(j\) la columna. La transpuesta de una matriz \(A\) se denota como \(A^T\).
\newline

Se denota mediante \(\lfloor x \rceil\) el redondeo de \(x\) al entero más cercano. Por ejemplo: \(\lfloor 2{,}3 \rceil = 2\), \(\lfloor 2{,}5 \rceil = 3\) y \(\lfloor 2{,}8 \rceil = 3\).
\newline

Se denota mediante $\left\| x\right\|$ al valor absoluto de x.
\newline

Para las reducciones modulares se emplean dos tipos: una centrada en cero y otra correspondiente a la reducción modular estándar. Para la reducción modular centrada en cero, sea \(\alpha\) un entero par. Esta operación se define como:
\begin{equation}
	r' = r \ \text{mod}^{\pm} \alpha \quad \Longrightarrow \quad -\dfrac{\alpha}{2} < r' \le \dfrac{\alpha}{2}
\end{equation}

Mientras que la reducción modular estándar se denota como:
\begin{equation}
	r' = r \ \text{mod}^{+} \alpha \quad \Longrightarrow \quad 0 \le r' < \alpha
\end{equation}

Finalmente, se denota mediante \(s \leftarrow S\) la selección de \(s\) de manera uniformemente aleatoria del conjunto \(S\). Si \(S\) representa una distribución de probabilidad, entonces \(s\) se selecciona de acuerdo con dicha distribución.
\newpage

\subsubsection{Transformada Teórica de Números o Number Theoretic Transform (\gls{ntt})}
Para acelerar las operaciones de multiplicación en el esquema basado en retículas, se utiliza la Transformada Teórica de Números (\gls{ntt}, por sus siglas en inglés), la cual permite reducir la complejidad temporal de la multiplicación de polinomios desde \(\mathcal{O}(n^2)\), correspondiente al método tradicional, hasta \(\mathcal{O}(n \log(n))\). Para más detalles, consúltese \cite{cryptoeprint:2024/585}.
\newline

Antes de pasar a explicar el funcionamiento de este método es relevante aclarar que se trabaja sobre el siguiente anillo de polinomios para realizar las operaciones, denotado mediante \(R_q\):
\begin{equation}
	R_q \coloneqq \dfrac{\mathbb{Z}_q[X]}{X^n + 1}
\end{equation}

En la implementación especificada de Kyber, según el artículo \cite{kyber-spec-2021}, se utiliza un valor de \(q=3329\) y \(n=256\). Esta elección es esencial para permitir el uso de la multiplicación mediante la Transformada Teórica de Números (\gls{ntt}), la cual requiere que \(n| (q-1)\), es decir, que \(n\) divida a \((q-1)\). Esta condición garantiza la existencia de \(n\) raíces enésimas de la unidad en \(Z_q\), lo cual es necesario para definir la \gls{ntt}. La validez de esta afirmación se fundamenta en el siguiente teorema \cite{moreno-roots}:
\newline

\begin{theorem}
	Para \(n\), \(q>1\), el cuerpo  \(Z_q\) tiene una raíz enésima de la unidad si y solo sí \(n| (q-1)\)
\end{theorem} 
\begin{proof}
	Si \(\omega\) es una raíz enésima de la unidad en el conjunto \( \mathbb{Z}_q \), entonces el conjunto:
	\begin{equation}
		\Omega = \left\{1, \omega, \omega^2, \dots, \omega^{n-1} \right\}
	\end{equation}
	forma un subgrupo cíclico \( H \) del grupo multiplicativo \( G_{q-1} \). Por el Teorema de Lagrange, se concluye que el orden de \( H \) divide al orden de \( G_{q-1} \), es decir, \( n \mid (q-1) \).
	\newline
	
	Dado que \( G_{q-1} \) es también un grupo cíclico, existe un generador \(\alpha\) tal que, por el pequeño teorema de Fermat, se cumple:
	\begin{equation}
		\alpha^{q-1} = 1
	\end{equation}
	
	Por lo tanto, el grupo \( G_{q-1} \) puede escribirse como:
	\begin{equation}
		G_{q-1} = \left\{1, \alpha, \alpha^2, \dots, \alpha^{q-2} \right\}
	\end{equation}
	
	Si se define \(\omega\) como:
	\begin{equation}
		\omega = \alpha^{\frac{q-1}{n}},
	\end{equation}
	
	entonces:
	\begin{equation}
		\omega^n = \alpha^{q-1} = 1,
	\end{equation}
	
	y además, para todo \( 0 < k < n \), se cumple:
	\begin{equation}
		k \cdot \frac{q-1}{n} < q-1 \quad \Rightarrow \quad \omega^k \neq 1.
	\end{equation}
	
	Por lo tanto, \(\omega\) es una raíz enésima de la unidad en \(\mathbb{Z}_q\).
\end{proof}
\newpage

Por tanto, el polinomio \(X^{256} + 1\) se puede factorizar sobre el cuerpo \(\mathbb{Z}_q\). En la implementación concreta del esquema Kyber, este polinomio se descompone en 128 factores cuadráticos.
\newline

A este polinomio se le aplica la transformada \gls{ntt} a sus coeficientes, la cual no es más que una variación de la \gls{dft} aplicada a cuerpos finitos \(\mathbb{Z}_q\) y aplicada a polinomios de grado \(n\). Para ello, se definen dos operaciones fundamentales:
\begin{enumerate}
	\item La transformada directa:
	\begin{equation}
		\hat{a}_j=\sum_{i=0}^{n-1} \phi^{i\left(2j+1\right)} a_j \ \text{mod} \ q
	\end{equation}
	\item  La transformada inversa:
	\begin{equation}
		a_i=n^{-1} \sum_{j=0}^{n-1} \phi^{-i\left(2j+1\right)} \hat{a}_j \ \text{mod} \ q
	\end{equation}
\end{enumerate}

Donde $\phi$ es un valor tal que $\phi^2=\omega$, con $\omega$ una raíz enésima de la unidad en \(\mathbb{Z}_q\) y \(n^{-1}\) es la inversa multiplicativa de \(n\) en \(\mathbb{Z}_q\).
\newline

A continuación, se presenta un ejemplo extraído de \cite{cryptoeprint:2024/585} para ilustrar su funcionamiento. Sea el polinomio \(G(x)=5+6x+7x^2+8x^3\), cuyo vector de coeficientes es \(g=[5,6,7,8]\). Trabajando en el anillo \(\mathbb{Z}_{7681}\), y tomando $\phi=1925$, se puede calcular la transformada \gls{ntt} \(\hat{g}\). Aplicando luego la transformada inversa, es posible recuperar el vector original \(g\).
\begin{equation}
	\hat{g}=\begin{bmatrix}
		\phi^0 & \phi^1 & \phi^2 & \phi^3\\
		\phi^0 & \phi^3 & \phi^6 & \phi^1\\
		\phi^0 & \phi^5 & \phi^2 & \phi^7\\
		\phi^0 & \phi^7 & \phi^6 & \phi^5\\
	\end{bmatrix} \cdot \begin{bmatrix}
	5\\
	6\\
	7\\
	8
	\end{bmatrix}=\begin{bmatrix}
	1 & 1925 & 3383 & 6468\\
	1 & 6468 & 4298 & 1925\\
	1 & 5756 & 3383 & 1213\\
	1 & 1213 & 4298 & 5756\\
	\end{bmatrix} \cdot \begin{bmatrix}
	5\\
	6\\
	7\\
	8 \end{bmatrix}= \begin{bmatrix}
	2489\\
	7489\\
	6478\\
	6607 \end{bmatrix}
\end{equation}

Aplicando la transformada inversa, donde la inversa de $\phi=1925$ en \(\mathbb{Z}_{7681}\) es $\phi^{-1}=1213$ y el inverso del orden del polinomio \(n=4\) es \(n^{-1}=5761\):
\begin{equation}
	g=n^{-1}\begin{bmatrix}
		\phi^0 & \phi^0 & \phi^0 & \phi^0\\
		\phi^{-1} & \phi^{-3} & \phi^{-5} & \phi^{-7}\\
		\phi^{-2} & \phi^{-6} & \phi^{-2} & \phi^{-6}\\
		\phi^{-3} & \phi^{-1} & \phi^{-7} & \phi^{-5}\\
	\end{bmatrix} \cdot \hat{g}=5761\begin{bmatrix}
	1 & 1 & 1 & 1\\
	1213& 5756& 6468& 1925\\
	4298& 3383& 4298& 3383\\
	5756& 1213& 1925& 6468\\
	\end{bmatrix} \cdot \begin{bmatrix}
	2489\\
	7489\\
	6478\\
	6607 \end{bmatrix}
\end{equation}

Con estas transformadas definidas se define el producto o convolución negativa en el anillo \(	R_q \coloneqq \dfrac{\mathbb{Z}_q[X]}{X^n + 1}\) entre dos polinomios \(g\) y \(h\) como:
\begin{equation}
	g\cdot h= \text{NTT}^{-1}(\text{NTT}(g)\circ\text{NTT}(h))
\end{equation}

Donde la operación \(\circ\) denota la multiplicación elemento a elemento (o punto a punto) entre los vectores en ${Z}_q[X]$.
\newpage

Ahora, queda demostrar que el tiempo de ejecución de la \gls{ntt} es de \(\mathcal{O}(n \log(n))\). Para lograr este cometido, se aprovechan dos propiedades que cumplen estas raíces calculadas:
\begin{enumerate}
	\item Peridicidad: 
	\begin{equation}
		\phi^{k+2n}=\phi^k
	\end{equation}
	\item Simetría:
	\begin{equation}
		\phi^{k+n}=\phi^{-k}
	\end{equation}
\end{enumerate}

Con estas propiedades, se puede implementar el algoritmo de Cooley-Tukey \cite{CooleyALG} que consiste en ir descomponiendo el problema en mitades de manera recursiva para reducir al máximo la cantidad de cálculos realizados. Partiendo de la transformada directa y desarrollando:
\begin{equation}
	\hat{a}_j=\sum_{i=0}^{n-1} \phi^{i\left(2j+1\right)} a_j \ \text{mod} \ q = \sum_{i=0}^{n/2-1} \phi^{4ij+2i} a_{2i} + \phi^{2j+1} \sum_{i=0}^{n/2-1} \phi^{4ij+2i} a_{2i+1} \ \text{mod} \ q
\end{equation}

Si se sustituye \(A_j=\sum_{i=0}^{n/2-1} \phi^{4ij+2i} a_{2i}\) y \(B_j=\sum_{i=0}^{n/2-1} \phi^{4ij+2i} a_{2i+1}\). Ahora aplicando simetría en \(\phi\):
\begin{equation}
	\hat{a}_j=A_j+\phi^{4ij+2i}B_j  \ \text{mod} \ q
\end{equation}
\begin{equation}
	\hat{a}_{j+n/2}=A_j-\phi^{4ij+2i}B_j  \ \text{mod} \ q
\end{equation}

Donde las matrices \(A_j\) y \(B_j\)pueden obtenerse como el resultado de aplicar la \gls{ntt} sobre la mitad de los puntos, gracias a la estructura recursiva del algoritmo. Esto implica que, si \(n\) es una potencia de \(2\), el proceso puede repetirse recursivamente sobre subproblemas de tamaño cada vez menor, hasta alcanzar el caso base. 
\newline

De manera similar, se puede mostrar esta propiedad para la transformada inversa. Por tanto, se demuestra que este algoritmo tiene complejidad \(\mathcal{O}(n \log(n))\).
\newline

En el caso concreto de Kyber \cite{kyber-spec-2021}, la Transformada (\gls{ntt}) de un polinomio en el anillo \(R_q\) se representa como un vector de 128 polinomios de grado 1.
\newline

Sean las 256 raíces enésimas de la unidad \(\left\{\xi, \xi^3, \dots, \xi^{255}\right\}\), con \(\xi = 17\) como primera raíz primitiva. Entonces, se cumple que:
\begin{equation}
	X^{256}+1=\prod_{i=0}^{127}\left(X^2- \xi^{2i+1}\right)
\end{equation}

Aplicando la \gls{ntt} propuesta en \cite{kyber-spec-2021} a un polinomio \(f\in R_q\) se obtiene:
\begin{equation}
	NTT(f)=\hat{f}=\left(\hat{f}_0+\hat{f}_1X,\hat{f}_2+\hat{f}_3X,...,\hat{f}_{254}+\hat{f}_{255}X\right)
\end{equation}

Con
\begin{equation}
	\begin{array}{l}
		\hat{f}_{2i}=\sum\limits_{j=0}^{127} f_{2j}\xi^{\left(2i+1\right)j}\\
		\hat{f}_{2i+1}=\sum\limits_{j=0}^{127} f_{2j+1}\xi^{\left(2i+1\right)j}
	\end{array}
\end{equation}

Donde mediante la transformada directa \gls{ntt} e inversa \gls{ntt}$^{-1}$ se puede realizar el producto de \(f\),\(g\in R_q\) de manera eficiente de la siguiente manera: 
\begin{equation}
	h=f\cdot g = \text{NTT}^{-1}\left[\text{NTT}(f) \circ \text{NTT}(g)\right]
\end{equation}

Siendo \(\hat{h}=\hat{f}\circ\hat{g}=\text{NTT}(f)\circ \text{NTT}(g)\) la multiplicación base definida como:
\begin{equation}
	\hat{h}_{2i}+\hat{h}_{2i+1}X=(\hat{f}_{2i}+\hat{f}_{2i+1})\left(\hat{g}_{2i}+\hat{g}_{2i+1}\right) \ \text{mod} \left(X^2-\xi^{2i+1}\right)
\end{equation}


En la figura \ref{fig:NTTFigure} se puede ver una representación gráfica de como funciona este proceso.
\begin{figure}[H]
	\centering
		\begin{circuitikz}
			\tikzstyle{every node}=[font=\LARGE]
			\draw  (5.25,13.75) rectangle (7.75,12.25);
			\draw  (13.5,13.75) rectangle (16,12.25);
			\node [font=\LARGE] at (6.5,13) {f $\cdot$ g};
			\node [font=\LARGE] at (14.75,13) {h};
			\draw [ color={rgb,255:red,255; green,40; blue,40}, ->, >=Stealth] (7.75,13) -- (13.5,13);
			\node [font=\LARGE, color={rgb,255:red,255; green,0; blue,0}] at (10.5,13.5) {$\mathcal{O}(n^2)$};
			\node [font=\normalsize] at (10.5,12.5) {Multiplicación directa};
			\draw  (5.25,9.5) rectangle (7.75,8);
			\draw  (13.5,9.5) rectangle (16,8);
			\draw [ color={rgb,255:red,0; green,64; blue,255}, ->, >=Stealth] (6.5,12.25) -- (6.5,9.5);
			\draw [ color={rgb,255:red,0; green,64; blue,255}, ->, >=Stealth] (7.75,8.75) -- (13.5,8.75);
			\draw [ color={rgb,255:red,0; green,64; blue,255}, ->, >=Stealth] (14.75,9.5) -- (14.75,12.25);
			\node [font=\LARGE] at (5.75,10.75) {NTT};
			\node [font=\LARGE] at (15.75,10.75) {NTT$^{-1}$};
			\node [font=\LARGE, color={rgb,255:red,0; green,64; blue,255}] at (10.5,9.25) {$\mathcal{O}(n\log(n))$};
			\node [font=\normalsize] at (10.75,8.25) {Multiplicación mediante NTT};
			\node [font=\LARGE] at (6.5,8.75) {$\hat{f}$ $\circ$ $\hat{g}$};
			\node [font=\LARGE] at (14.75,8.75) {$\hat{h}$};
		\end{circuitikz}
	\label{fig:NTTFigure}
	\caption{Representación del cálculo de un producto de polinomios mediante \gls{ntt} en comparación con los algoritmos clásicos.}
\end{figure}


\subsubsection{Aprendizaje Con Errores o Learning With Errors (\gls{lwe})}
Para la elaboración de esta sección, se ha utilizado la información contenida en el artículo \cite{LWElearning}, el cual expone los fundamentos matemáticos del problema de Aprendizaje con Errores (Learning With Errors, \gls{lwe}). Este problema constituye la base teórica sobre la que se sustenta el esquema criptográfico Kyber.
\newline

Para ello, los esquemas basados en \gls{lwe} trabajan con un objeto matemático conocido como retícula. Una retícula es un conjunto discreto de puntos en el espacio, generado por todas las combinaciones lineales con coeficientes enteros de un conjunto de \(n\) vectores linealmente independientes que conforman una base $\mathbb{B}=\left\{b_1,...,b_n\right\}$:
\begin{equation}
	\mathcal{A}=\mathcal{L}(\mathbb{B})= \left\{\sum_{i\in n} z_ib_i: z\in Z^n\right\}
\end{equation}

Esta definición será útil en la demostración de la seguridad de los esquemas basados en \gls{lwe}. Aun así, antes de pasar a la explicación del algoritmo de intercambio de claves públicas, es necesario definir el problema de Aprendizaje con Errores.
\newline

El Aprendizaje con Errores consiste en la tarea de distinguir entre parejas de la forma \((a_i, b_i) \in \mathbb{Z}_q^n \times \mathbb{Z}_q\), donde \(b_i \approx \langle a_i, s \rangle\), y parejas generados de manera completamente aleatoria. Donde, \(s \in \mathbb{Z}_q^n\) es el secreto  que se mantiene fijo para todas las muestras, \(a_i \in \mathbb{Z}_q^n\) es un vector elegido uniformemente al azar, y \(\langle ,  \rangle\) denota el producto escalar Euclídeo.
\newline

La principal utilidad criptográfica del problema \gls{lwe} radica en que, para quien no conoce el secreto, resulta computacionalmente difícil distinguir entre pares estructurados y pares aleatorios. En cambio, quien sí conoce el secreto puede identificar fácilmente qué parejas son consistentes con este, lo que permite construir esquemas seguros de cifrado e intercambio de claves.
\paragraph{\gls{lwe} en anillos}
\mbox{}\\

En Kyber no se parte del problema LWE “puro”, ya que los esquemas basados directamente en \gls{lwe} suelen presentar claves de tamaño y tiempos de cálculo con un orden de magnitud \(\mathcal{O}(n^2)\) \cite{Micciancio2009}. Esto se debe a que, al trabajar con matrices genéricas, se carece de la estructura de anillo necesaria para aprovechar multiplicaciones mediante convolución rápida, es decir, la \gls{ntt}. En cambio, Kyber se fundamenta en una variante conocida como Modulus-LWE (\gls{mlwe}), donde las operaciones de multiplicación matricial se pueden implementar como multiplicaciones de polinomios en $\dfrac{\mathbb{Z}_q[x]}{X^n + 1}$, lo cual permite usar la \gls{ntt}.
\newline


Antes de describir en detalle el funcionamiento de la \gls{mlwe}, conviene introducir primero el problema Ring-LWE (\gls{rlwe}). La \gls{mlwe} puede entenderse como una extensión modular (un vector) de \gls{rlwe}. Este enfoque permite romper la simetría inherente a un anillo y aporta mayor flexibilidad en la selección de parámetros para lograr una mejor relación entre eficiencia y resistencia criptográfica \cite{cryptoeprint:2019/930}.
\newline

En cuanto a la eficiencia en la reducción del tamaño de las claves, esta puede observarse claramente a través del siguiente ejemplo ilustrativo:
\begin{table}[H]
	\centering
	\renewcommand{\arraystretch}{1.2}
	\begin{tabular}{lcc}
		\hline
		Esquema & \gls{lwe} & \gls{rlwe} \\
		\hline
		Tamaño clave pública & \(\mathcal{O}(n^2)\) bits & \(\mathcal{O}(n)\)  \\
		\hline
		Operación para generar la pareja & \(b_i=\langle a_i, s \rangle + e_i\) & \(b[x]=a[x]\cdot s[x]+e[x]\)\\
		\hline
		Matriz \(A\) &$\begin{pmatrix}
							1  &2\\
							3 &4
						\end{pmatrix}$ &3+11X\\
		\hline
		Secreto \(s\) & $\begin{pmatrix}
							5\\
							6
						\end{pmatrix}$&5+6X\\
		\hline
		Error \(e\)& $\begin{pmatrix}
							1\\
							1
						\end{pmatrix}$&1+X\\
		\hline
		Resultado \(b\)&$\begin{pmatrix}
							1\\
							6
						\end{pmatrix}$ & 1+6X\\
		\hline
	\end{tabular}
	\caption{Tabla con un ejemplo numérico del calculo del valor \(b\) necesario para el problema \gls{lwe} como en \gls{rlwe} para ver como efectivamente en la clave pública \((a,b)\) el tamaño es menor. En este ejemplo se trabaja en \(\mathbb{Z}_{17}\) y \(X^2 + 1\). En \gls{mlwe} las claves son de tamaño similar que en \gls{rlwe} pues la matriz \(A\) se puede reconstruir a través de una semilla y la matriz \(b\) solo ve reescalada en función de la seguridad empleada.}
	\label{tab:LWE-RLWE}
\end{table}


\newpage

\paragraph{Aplicación del \gls{rlwe} para el intercambio de claves públicas}
\mbox{}\\

A continuación, se presenta el algoritmo principal empleado para el intercambio de claves basado en el problema \gls{rlwe} que fundamenta matemáticamente el funcionamiento de Kyber. El uso de la \gls{mlwe} es similar y se puede consultar en la sección \ref{sub:kybAlg}.
\newline

El algoritmo de generación de claves emplea los valores \(q\) y \(n\) , que definen la estructura algebraica, para calcular la llave privada \(sk\)  y la llave pública \(pk\).
\begin{algorithm}[H]
	\caption{Generación claves \gls{rlwe} en $\mathbb{Z}_q[x]/(X^n + 1)$ \protect\footnotemark[\value{footnote}]} 
	$\begin{array}{p{\textwidth}}
		\textbf{Entrada: } \(q\), \(n\)\\ 
		\hline
		\textbf{Salida: } \(sk\), \(pk\) \\ 
		\hline
	\end{array}$
	\begin{algorithmic}[1]
		\State Generar \(a\in R_q\)  al azar según una distribución uniforme.
		\State Generar \(sk, e \in R_q\) como elementos pequeños según la distribución del error. \Statex \Comment{Binomial en Kyber, discreta Gaussiana en otros esquemas}
		\State Calcular 
		\begin{equation}
			b:=a\cdot sk + e
		\end{equation}
		\State \Return \((sk, pk:=a||b)\)
	\end{algorithmic}
\end{algorithm}

En el algoritmo de cifrado a partir de la llave pública, \(pk\) y un mensaje \(z\in \{0,1\}^n \) se obtienen los textos cifrados \(u\) y \(v\) que serán enviados al poseedor de la clave privada para descifrar el mensaje. 
\begin{algorithm}[H]
	\caption{Cifrado \gls{rlwe} \protect\footnotemark[\value{footnote}]} 
	$\begin{array}{p{\textwidth}}
		\textbf{Entrada: } \(pk\), \(z\), \(q\), \(n\)\\ 
		\hline
		\textbf{Salida: } \(u\), \(v\) \\ 
		\hline
	\end{array}$
	\begin{algorithmic}[1]
		\State Generar \(r, e_1, e_2 \in R_q\) como elementos pequeños según la distribución del error.
		\State Calcular el primer texto cifrado \(u\):
		\begin{equation}
			u:=a\cdot r + e_1 \ \text{mod}^{+}\text{q}
		\end{equation}
		\State Calcular el segundo texto cifrado \(v\):
		\begin{equation}
			v:=b\cdot r+e_2+\left\lfloor \dfrac{q}{2} \right\rceil \cdot z \ \text{mod}^{+}\text{q}
		\end{equation}
		\State \Return \((u,v)\) 
		 
	\end{algorithmic}
\end{algorithm}

En el algoritmo de descifrado, a partir de los textos cifrados \(u\) y \(v\)se recupera el mensaje original \(z\). La seguridad del esquema radica en el término de ruido ($\varepsilon=r\cdot e- s\cdot e_1 +e_2$) que ofusca el mensaje, el cual solo puede recuperarse correctamente usando la clave secreta siempre que \(\left\|\varepsilon\right\|< q/4\).
\newline

Para mantener este error dentro de niveles aceptables, es fundamental ajustar adecuadamente los parámetros de las distribuciones de ruido en el algoritmo Kyber. En la versión empleada en este trabajo (Kyber1024), esto se traduce en una tasa de error de $2^{-174}$ \cite{pqcrystalssecurity}.
\footnotetext{
	En Kyber se usa la variante \gls{mlwe}, donde
	\(s\in R_q^d\), \(A\in R_q^{d\times d}\) y \(b=A\cdot s + e \in R_q^d\);
	el pseudocódigo completo en \gls{mlwe} aparece en la sección \ref{sub:kybAlg}.
}
\newpage
\begin{algorithm}[H]
	\caption{Descifrado \gls{rlwe} \protect\footnotemark[\value{footnote}]} 
	$\begin{array}{p{\textwidth}}
		\textbf{Entrada: } \(sk\), \(u\), \(v\), \(q\), \(n\)\\ 
		\hline
		\textbf{Salida: } \(z\)\\ 
		\hline
	\end{array}$
	\begin{algorithmic}[1]
		\State Calcular el polinomio \(z'\) a partir del cual se calcula el mensaje.
		\begin{equation}
			\label{eq:ruidoLWE}
			z':=v-u\cdot s = \left(r\cdot e- s\cdot e_1 +e_2\right)+\left\lfloor \dfrac{q}{2} \right\rceil \cdot z \ \text{mod}^{+}\text{q}
		\end{equation}
		\State Calcular la distancia de cada coeficiente \((z'_i)\) de \(z'\):
		\Statex \hspace{1em} \textbf{1.1:} Distancia a \(0\):
		\begin{equation}
			\text{d}_i(0):=\left\| z'_i \ \text{mod}^{\pm} \left\lfloor \dfrac{q}{2} \right\rceil \right\| 
		\end{equation} 
		\Statex \hspace{1em} \textbf{1.2:} Distancia a $\left\lfloor \dfrac{q}{2} \right\rceil$:
			\begin{equation}
			\text{d}_i\left(\dfrac{q}{2}\right)=\left\| \left\lfloor \dfrac{q}{2} \right\rceil - \text{d}_i(0)\right\| 
		\end{equation} 
		\If{$\text{d}_i(0)<\text{d}_i\left(\dfrac{q}{2}\right)$}
			\Statex
			\State Descifrar el bit \(i\) del mensaje \(z\) como un \(0\).
		\Else
			\State Descifrar el bit \(i\) del mensaje \(z\) como un \(1\).
		\EndIf
		\State \Return \(z\)
	\end{algorithmic}
\end{algorithm}
 \footnotetext{
 	En Kyber se usa la variante \gls{mlwe}, donde
 	\(s\in R_q^d\), \(A\in R_q^{d\times d}\) y \(b=A\cdot s + e \in R_q^d\);
 	el pseudocódigo completo en \gls{mlwe} aparece en la sección \ref{sub:kybAlg}.
 }
 En el anexo \ref{chap:ej:rlwe} se puede ver un breve ejemplo de aplicación numérica de funcionamiento de los algoritmos anteriores.

\paragraph{Uso de \gls{mlwe} en Kyber}
\mbox{}\\
A partir del siguiente articulo se define la \gls{mlwe} que es el esquema de fondo en Kyber \cite{cryptoeprint:2019/930}.
\newline

Sean \(R_q\) el anillo a trabajar y \(d\) la dimensión de la retícula. Se define:
\[
\begin{aligned}
	s &\in R_q^d 
	&& \quad \text{(el vector secreto)}\\
	A &\in R_q^{d\times d} 
	&& \quad \text{(la matriz pública, muestreada \(R_q^{d\times d}\) a partir de una semilla $\rho$)}\\
	e &\in R_q^d 
	&& \quad \text{(el vector de error, con coeficientes “pequeños”)}\\
	b &\in R_q^d 
	&& \quad \text{(la pareja resultante)}\\
	b &= A \cdot s + e 
	&& \in R_q^d
\end{aligned}
\]
Entonces, el problema \gls{mlwe} (como extensión modular de \gls{rlwe}) se define como el problema de distinguir muestras \((A, b)\) de parejas uniformes en \(R_q^{d\times d}\times R_q^d\).
\newline

Como se observa, esta definición resulta relativamente sencilla de entender a partir del ejemplo anterior. Sin embargo, las razones que avalan su seguridad son diferentes: al presentar una estructura menos uniforme, la \gls{mlwe} ofrece mejores garantías frente a ataques que explotan la regularidad de los anillos. 
\newline

Por tanto, la \gls{mlwe} es un punto intermedio entre ambos esquemas que permite:
\begin{itemize}
	\item Escalabilidad en seguridad mediante el parámetro $d$.
	\item Mejor eficiencia que en \gls{lwe}, al poder emplear la \gls{ntt} sobre anillos.
	\item Reducción del tamaño de claves, pues la matriz \(A\) puede reconstruirse a partir de una semilla.
	\item Mayor robustez, dado que su estructura “menos simétrica” que un anillo puro dificulta algunos ataques específicos a \gls{rlwe}.
\end{itemize}


\subsubsection{Fundamentos de seguridad de Kyber}
\paragraph{Seguridad esperada}
\mbox{}\\
\paragraph{Análisis respecto a ataques conocidos}
\mbox{}\\
\subsubsection{Parámetros Kyber empleados}
A continuación, se presenta una tabla con los parámetros del esquema Kyber1024 empleado en este trabajo fin de grado.
\begin{table}[h]
	\centering
	\renewcommand{\arraystretch}{1.2}
	\begin{tabular}{lcccccccc}
		\hline
		&\(n\)&\(k\)&\(q\)&\(\eta_1\)&\(\eta_2\)&\(d_u\)&\(d_v\)&\(\delta\)\\
		\hline
		Kyber1024&\(256\)&\(4\)&\(3329\)&\(2\)&\(2\)&\(11\)&\(5\)&\(2^{-174}\)\\
		\hline
	\end{tabular}
	\caption{Tabla con los parámetros utilizados por Kyber1024. El valor de \(n=256\) se elige para permitir una escalabilidad sencilla y permitir distintos niveles de seguridad sin perder capacidad de tener un nivel de ruido aceptable. El valor de \(k=4\) fija el tamaño de la retícula e implica una seguridad de 256 bits. El valor de \(q=3329\) es un primo que satisface \(n|(q-1)\) para permitir la \gls{ntt}, los primos anterior y siguiente que también satisfacen esta propiedad conllevan probabilidades de fallo demasiado altas. Los valores $\eta_1$ y $\eta_2$ definen el ruido y junto a \(d_u\) y \(d_v\) se usan para equilibrar la seguridad y la tasa de fallos $\delta$.}
	\label{tab:KyberParams}
\end{table}


\newpage

\subsubsection{Algoritmos principales \cite{kyber-spec-2021}}
\label{sub:kybAlg}
El algoritmo de generación de llaves genera las llaves pública \((pk)\) y privada \((sk)\) a partir de los parámetros de la tabla \ref{tab:KyberParams}.

\begin{itemize}
	\item La función $\text{Parse}(x)$ se encarga de convertir cadenas de bits a su representación \gls{ntt} garantizando que los coefientes \(a_i\) sean del tamaño adecuado \(\log_2(q)\approx 11.7\) y no permitiendo desbordamientos \(a_i<q\).
	\item La función $\text{CBD}_\eta(x)$ muestrea el ruido a partir mediante una distribución binomial. Convierte un vector de bits \(B\in\mathcal{B}^{64\eta}\) a un polinomio \(f\in R_q\).
	\item La función $\text{Encode}_k(x)$ convierte de un vector de bits \(B\in\mathcal{B}^{32l}\) a un polinomio \(f\in R_q\).
\end{itemize}
  

\begin{algorithm}[H]
	\caption{Generación llaves en Kyber}
	$\begin{array}{p{\textwidth}}
		\textbf{Salida: } \(pk\in \mathcal{B}^{12\cdot k\cdot n/8+32}\), \(sk\in \mathcal{B}^{24\cdot k\cdot n/8+96}\) \\ 
		\hline
	\end{array}$
	\begin{algorithmic}[1]
		\State Obtener \(d, z\in\mathcal{B}^{32}\) de manera aleatoria usando una distribución uniforme.
		\State Obtener dos nuevos parámetros \(\rho, \sigma\) expandiendo el valor inicial \(d\):
		\begin{equation}
			(\rho, \sigma):=\text{SHA3-512}(d)
		\end{equation}
		\Statex Se crean las matrices para realizar las operaciones de los algoritmos de la  \gls{rlwe}.
		\For{i=0:k-1}
			\For{j=0:k-1}
				\State $\hat{A}[i][j]:=\text{Parse}[\text{SHAKE-128}(\rho,j,i)]$ \Comment{Muestreo matriz en el dominio NTT.}
			\EndFor
		\EndFor
		\State $N:=0$
		\For{i=0:k-1}
			\State $s[i]:= \text{CBD}_{\eta_1}[\text{SHAKE-256}(\sigma,N)]$ \Comment{Muestreo secreto.}
			\State $N:= N+1$
		\EndFor
		\For{i=0:k-1}
			\State $e[i]:= \text{CBD}_{\eta_1}[\text{SHAKE-256}(\sigma,N)]$ \Comment{Muestreo error.}
			\State $N:= N+1$
		\EndFor
		\State Se convierten las magnitudes mediante la NTT para agilizar los cálculos:
		\begin{equation}
				\begin{array}{l}
					\hat{s}:=\text{NTT}(s)\\
					\hat{e}:=\text{NTT}(e)\\
					\hat{t}:=\hat{A}\circ\hat{s}+\hat{e}\\
				\end{array} 	
		\end{equation}
		\State Se calcula la llave pública:
		\begin{equation}
			pk:=\text{Encode}_{12}(\hat{t}\ \text{mod}^{+}\text{q} )||\rho 
		\end{equation}
		\Statex \Comment{Se envía $\hat{b}$ junto con la semilla $\rho$ para el calculo de $\hat{A}$.}
		\Statex \Comment{Así se reduce el tamaño de la clave enviada.}
		\State Se calcula la llave secreta:
		\begin{equation}
			sk:=\text{Encode}_{12}(\hat{s}\ \text{mod}^{+}\text{q} )||pk||\text{SHA3-256}(pk)||z
		\end{equation}
		\Statex \Comment{Se realiza esta concatenación para cumplir la seguridad \gls{cca2} mediante la \gls{tfo}.}
		\State \Return $(pk,sk)$
	\end{algorithmic}
\end{algorithm}
\newpage

En el algoritmo de cifrado Kyber se obtiene el texto cifrado \(c\) a partir de la llave pública \(pk\), un mensaje \(m\) y una semilla aleatoria \(\gamma\) mediante el uso de la \gls{ntt}.
\begin{itemize}
	\item Las funciones $\text{Compress}_q(x,y)$ y $\text{Decompress}_q(x,y)$ se usan para reducir el tamaño de los textos cifrados basándose en el fundamento descrito para los mecanismos basados en la \gls{rlwe}. 
	\item La función $\text{Decode}_k(x)$  convierte de un polinomio \(f\in R_q\) a un vector de bits \(B\in\mathcal{B}^{32l}\).
\end{itemize}

\begin{algorithm}[H]
	\caption{Cifrado Kyber}
	$\begin{array}{p{\textwidth}}
		\textbf{Entrada: } \(pk\in \mathcal{B}^{12\cdot k\cdot n/8+32}\), \(m\in \mathcal{B}^{32}\), \(\gamma\in \mathcal{B}^{32}\)\\ 
		\hline
		\textbf{Salida: } \(c\in \mathcal{B}^{d_u\cdot k\cdot n/8+d_v\cdot n/8}\)\\ 
		\hline
	\end{array}$
	\begin{algorithmic}[1]
		\State Calcular los parámetros necesarios:
		\State \(N:=0\)
		\For{i=0:k-1}
		\State $r[i]:= \text{CBD}_{\eta_1}[\text{SHAKE-256}(\gamma,N)]$ \Comment{Muestreo elemento \(r\).}
		\State $N:= N+1$
		\EndFor
		\For{i=0:k-1}
		\State $e_1[i]:= \text{CBD}_{\eta_2}[\text{SHAKE-256}(\gamma,N)]$ \Comment{Muestreo del primer error.}
		\State $N:= N+1$
		\EndFor
		\State \(e_2[i]:= \text{CBD}_{\eta_2}[\text{SHAKE-256}(\gamma,N)]\) \Comment{Muestreo del segundo error.}
		\State Calcular los valores de los textos cifrados:
		\begin{equation}
			\begin{array}{l}
				\hat{r}:=\text{NTT}(r)\\
				u:= \text{NTT}^{-1}(\hat{A}^T\circ \hat{r})+e_1\\
				v:=\text{NTT}^{-1}(\hat{t}^T\circ \hat{r})+e_2+ \text{Decompress}_q[\text{Decode}_1(m),1]\\
				c_1:=\text{Encode}_{d_u}[\text{Compress}_q(u,d_u)]\\
				c_2:=\text{Encode}_{d_v}[\text{Compress}_q(v,d_v)]
			\end{array}
		\end{equation}
		\State \Return (\(c:=c1||c2\))
	\end{algorithmic}
\end{algorithm}
\newpage

 En el algoritmo de encapsulado Kyber a partir de la llave pública \(pk\) se obtiene el texto cifrado \(c\) y el secreto compartido \(k\). 
\begin{algorithm}[H]
	\small 
	\caption{Encapsulado Kyber}
	$\begin{array}{p{\textwidth}}
		\textbf{Entrada: } \(pk\in \mathcal{B}^{12\cdot k\cdot n/8+32}\)\\ 
		\hline
		\textbf{Salida: } \(c\in \mathcal{B}^{d_u\cdot k\cdot n/8+d_v\cdot n/8}\), \(k\in \mathcal{B}^{*}\)\\ 
		\hline
	\end{array}$
	\begin{algorithmic}[1]
		\State Obtener los valores necesarios a partir de la llave pública:
		\begin{equation}
			\begin{array}{l}
			\hat{t}:=\text{Decode}_{12}(pk)\\
			p:=pk+12\cdot k\cdot n/8
			\end{array} 
		\end{equation}
		\State Calcular la matriz \(\hat{A}^T\) a partir de \(\rho\) codificado en la llave pública.
		\State Obtener \(m'\) de manera aleatoria usando una distribución uniforme.
		\State Obtener los parámetros \(m, \kappa, \gamma\) a partir de \(m'\) y la llave pública:
		\begin{equation}
			\begin{array}{l}
				m:=\text{SHA3-256}(m')\\
				(\kappa,\gamma):=\text{SHA3-512}[m||\text{SHA3-256}(pk)]
			\end{array} 
		\end{equation}
		\State Obtener el texto cifrado:
		\begin{equation}
			c \gets \text{Cifrado Kyber}(pk,m,\gamma)
		\end{equation}
		\State Calcular el secreto compartido:
		\begin{equation}
			k:= \text{SHAKE-256}[\kappa||\text{SHA3-256}(c)]
		\end{equation}
		\State \Return (\(c,k\))
	\end{algorithmic}
\end{algorithm}

En el algoritmo de decapsulado Kyber a partir del texto cifrado \(c\)  y la llave secreta \(sk\) se puede obtener el secreto compartido \(k\).
\begin{algorithm}[H]
	\small
	\caption{Decapsulado Kyber}
	$\begin{array}{p{\textwidth}}
		\textbf{Entrada: } \(c\in \mathcal{B}^{d_u\cdot k\cdot n/8+d_v\cdot n/8}\), \(sk\in \mathcal{B}^{24\cdot k\cdot n/8+96}\)\\ 
		\hline
		\textbf{Salida: } \(k\in \mathcal{B}^{*}\)\\ 
		\hline
	\end{array}$
	\begin{algorithmic}[1]
		\State Obtener los valores descomprimidos \(u\), \(v\) y el valor de la llave secreta \(s\) en el dominio NTT:
		\begin{equation}
			\begin{array}{l}
				u:=\text{Decompress}_q[\text{Decode}_{d_u}(c,d_u)]\\
				v:=\text{Decompress}_q[\text{Decode}_{d_v}(c+d_u\cdot k\cdot n/8,d_v)]\\
				\hat{s}:=\text{Decode}_{12}(sk)
			\end{array}
		\end{equation}
		\State Obtener el mensaje \(m'\) cifrado anteriormente:
		\begin{equation}
			m':=\text{Encode}_1[\text{Compress}_q\left(v-\text{NTT}^{-1}(\hat{s}^T\circ \text{NTT}(u)),1\right)]
		\end{equation}
		\Statex Para Garantizar la seguridad ante ataques de canal lateral se vuelve a calcular el texto cifrado.
		\State Cifrar \(m'\) con la llave pública \(pk\) y el parámetro \(\gamma\) para obtener el texto cifrado \(c'\):
		\begin{equation}
			\begin{array}{l}
				pk:=sk+12\cdot k\cdot n/8\\
				h:=sk+24\cdot k\cdot n/8+32\\
				(\kappa, \gamma):=\text{SHA3-512}[m'||h]\\
				c'\gets \text{Cifrado Kyber}(pk,m',\gamma)
			\end{array}
		\end{equation}
		\State Comparar los textos cifrados obtenidos, añadiendo un nuevo parámetro \(z\) para textos cifrados no válidos.
		\begin{equation}
			z:=sk+24\cdot k \cdot n/8 +64
		\end{equation}
		\If{$c==c'$}
		\State \Return \(K:=\text{SHAKE-256}[\kappa||\text{SHA3-256}(c)]\) \Comment{Mismo secreto compartido.}
		\Else
		\State \Return \(K:=\text{SHAKE-256}[z||\text{SHA3-256}(c)]\) \Comment{No distinguible de llaves válidas.}
		\EndIf
	\end{algorithmic}
\end{algorithm}
\newpage

\subsection{SABER}
Para \cite{saber-spec-2020}
\subsection{Hamming Quasi-Cyclic (HQC)}
Para  \cite{hqc-spec-2022}


