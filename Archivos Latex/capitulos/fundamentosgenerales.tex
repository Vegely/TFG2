\chapter{Fundamentos generales}
En este capítulo se desarrollan las bases matemáticas de los distintos algoritmos a implementar.
\section{Introducción}
\section{Algoritmos de Hashing y Funciones de Salida Extendida\cite{FIPS202}}
\section{Métodos clásicos de cifrado asimétrico}
\subsection{RSA}
\subsection{ECC}
\subsection{Algoritmo de Shore}
\newpage
\section{Funcionamiento básico de los algoritmos postcuánticos}
En esta sección se describe el funcionamiento de los algoritmos postcuánticos analizados en este trabajo. Dado que no se desarrollaron implementaciones propias, sino que se utilizó el código proporcionado por el NIST en la tercera \cite{nistPQCround3} y cuarta \cite{nistPQCround4} ronda del proceso de estandarización, resulta apropiado presentar su funcionamiento aquí en lugar de en la sección de desarrollo.
\subsection{CRYSTALS-Kyber }
Para \cite{kyber-spec-2021}
\subsection{SABER}
Para \cite{saber-spec-2020}
\subsection{Hamming Quasi-Cyclic (HQC)}
Para  \cite{hqc-spec-2022}
\subsection{Bit Flipping Key Encapsulation (Bike)}
Para \cite{bike-spec-2022}
\newpage
\section{Fundamentos de seguridad de los algoritmos}
\subsection{CRYSTALS-Kyber }
Para \cite{kyber-spec-2021}
\subsection{SABER}
Para \cite{saber-spec-2020}
\subsection{Hamming Quasi-Cyclic (HQC)}
Para  \cite{hqc-spec-2022}
\subsection{Bit Flipping Key Encapsulation (Bike)}
Para \cite{bike-spec-2022}