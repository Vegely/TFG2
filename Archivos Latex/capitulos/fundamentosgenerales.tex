\chapter{Fundamentos generales}
En este capítulo se desarrollan las bases matemáticas de los distintos algoritmos a implementar.
\section{Introducción}
\section{Algoritmos de Hashing y Funciones de Salida Extendida\cite{FIPS202}}
\section{Métodos clásicos de cifrado asimétrico}
\subsection{RSA}
\subsection{ECC}
\subsection{Algoritmo de Shore}
\newpage
\section{Funcionamiento básico de los algoritmos postcuánticos}
En esta sección se describe el funcionamiento de los algoritmos postcuánticos analizados en este trabajo. Dado que no se desarrollaron implementaciones propias, sino que se utilizó el código proporcionado por el NIST en la tercera \cite{nistPQCround3} y cuarta \cite{nistPQCround4} ronda del proceso de estandarización, resulta apropiado presentar su funcionamiento aquí en lugar de en la sección de desarrollo.
\subsection{CRYSTALS-Kyber }
Se utilizará la misma notación empleada en el artículo \cite{kyber-spec-2021}. El conjunto de los enteros sin signo de 8 bits se denota por \(\mathcal{B} = \{0, \dots, 255\}\). Para representar vectores de tamaño \(k\), se utiliza la notación \(\mathcal{B}^k\), mientras que para vectores de tamaño arbitrario se emplea \(\mathcal{B}^*\).
\newline

Para trabajar con estos vectores, se utiliza el símbolo \(||\) para denotar la concatenación de dos cadenas, y la notación \(+k\) para indicar el desplazamiento de \(k\) bytes desde el inicio de una cadena. Por ejemplo, si se tiene una cadena \(a\) de longitud \(l\) y se concatena con una cadena \(b\), se obtiene:
\begin{equation}
	c = a || b
\end{equation}

Entonces:
\begin{equation}
	b = c + l
\end{equation}

Para denotar vectores, se utiliza la notación \(v[i]\), donde \(v\) es un vector columna e \(i\) indica la posición del elemento (empezando desde 0, si no se indica lo contrario). Para las matrices, se emplea la notación \(A[i][j]\), donde \(i\) representa la fila y \(j\) la columna. La transpuesta de una matriz \(A\) se denota como \(A^T\).
\newline

Se denota mediante \(\lfloor x \rceil\) el redondeo de \(x\) al entero más cercano. Por ejemplo: \(\lfloor 2{,}3 \rceil = 2\), \(\lfloor 2{,}5 \rceil = 3\) y \(\lfloor 2{,}8 \rceil = 3\).
\newline

Para las reducciones modulares se emplean dos tipos: una centrada en cero y otra correspondiente a la reducción modular estándar. Para la reducción modular centrada en cero, sea \(\alpha\) un entero par. Esta operación se define como:
\begin{equation}
	r' = r \ \text{mod}^{\pm} \alpha \quad \Longrightarrow \quad -\dfrac{\alpha}{2} < r' \le \dfrac{\alpha}{2}
\end{equation}

Mientras que la reducción modular estándar se denota como:
\begin{equation}
	r' = r \ \text{mod}^{+} \alpha \quad \Longrightarrow \quad 0 \le r' < \alpha
\end{equation}

Finalmente, se denota mediante \(s \leftarrow S\) la selección de \(s\) de manera uniformemente aleatoria del conjunto \(S\). Si \(S\) representa una distribución de probabilidad, entonces \(s\) se selecciona de acuerdo con dicha distribución.
\newpage

\subsubsection{Transformada Teórica de Números o Number Theoretic Transform (\gls{ntt})}
Para acelerar las operaciones de multiplicación en el esquema basado en retículas, se utiliza la Transformada Teórica de Números (\gls{ntt}, por sus siglas en inglés), la cual permite reducir la complejidad temporal de la multiplicación de polinomios desde \(\mathcal{O}(n^2)\), correspondiente al método tradicional, hasta \(\mathcal{O}(n \log(n))\). Para más detalles, consúltese \cite{cryptoeprint:2024/585}.
\newline

Antes de pasar a explicar el funcionamiento de este método es relevante aclarar que se trabaja sobre el siguiente anillo de polinomios para realizar las operaciones, denotado mediante \(R_q\):
\begin{equation}
	R_q \coloneqq \dfrac{\mathbb{Z}_q[X]}{X^n + 1}
\end{equation}

En la implementación especificada de Kyber, según el artículo \cite{kyber-spec-2021}, se utiliza un valor de \(q=3329\) y \(n=256\). Esta elección es esencial para permitir el uso de la multiplicación mediante la Transformada Teórica de Números (\gls{ntt}), la cual requiere que \(n| (q-1)\), es decir, que \(n\) divida a \((q-1)\). Esta condición garantiza la existencia de \(n\) raíces enésimas de la unidad en \(Z_q\), lo cual es necesario para definir la \gls{ntt}. La validez de esta afirmación se fundamenta en el siguiente teorema \cite{moreno-roots}:
\newline

\begin{theorem}
	Para \(n\), \(q>1\), el cuerpo  \(Z_q\) tiene una raíz enésima de la unidad si y solo sí \(n| (q-1)\)
\end{theorem} 
\begin{proof}
	Si \(\omega\) es una raíz enésima de la unidad en el conjunto \( \mathbb{Z}_q \), entonces el conjunto:
	\begin{equation}
		\Omega = \left\{1, \omega, \omega^2, \dots, \omega^{n-1} \right\}
	\end{equation}
	forma un subgrupo cíclico \( H \) del grupo multiplicativo \( G_{q-1} \). Por el Teorema de Lagrange, se concluye que el orden de \( H \) divide al orden de \( G_{q-1} \), es decir, \( n \mid (q-1) \).
	\newline
	
	Dado que \( G_{q-1} \) es también un grupo cíclico, existe un generador \(\alpha\) tal que, por el pequeño teorema de Fermat, se cumple:
	\begin{equation}
		\alpha^{q-1} = 1
	\end{equation}
	
	Por lo tanto, el grupo \( G_{q-1} \) puede escribirse como:
	\begin{equation}
		G_{q-1} = \left\{1, \alpha, \alpha^2, \dots, \alpha^{q-2} \right\}
	\end{equation}
	
	Si se define \(\omega\) como:
	\begin{equation}
		\omega = \alpha^{\frac{q-1}{n}},
	\end{equation}
	
	entonces:
	\begin{equation}
		\omega^n = \alpha^{q-1} = 1,
	\end{equation}
	
	y además, para todo \( 0 < k < n \), se cumple:
	\begin{equation}
		k \cdot \frac{q-1}{n} < q-1 \quad \Rightarrow \quad \omega^k \neq 1.
	\end{equation}
	
	Por lo tanto, \(\omega\) es una raíz enésima de la unidad en \(\mathbb{Z}_q\).
\end{proof}
\newpage

Por tanto, el polinomio \(X^{256} + 1\) se puede factorizar sobre el cuerpo \(\mathbb{Z}_q\). En la implementación concreta del esquema Kyber, este polinomio se descompone en 128 factores cuadráticos.
\newline

A este polinomio se le aplica la transformada \gls{ntt} a sus coeficientes, la cual no es más que una variación de la \gls{dft} aplicada a cuerpos finitos \(\mathbb{Z}_q\) y aplicada a polinomios de grado \(n\). Para ello, se definen dos operaciones fundamentales:
\begin{enumerate}
	\item La transformada directa:
	\begin{equation}
		\hat{a}_j=\sum_{i=0}^{n-1} \phi^{i\left(2j+1\right)} a_j \ \text{mod} \ q
	\end{equation}
	\item  La transformada inversa:
	\begin{equation}
		a_i=n^{-1} \sum_{j=0}^{n-1} \phi^{-i\left(2j+1\right)} \hat{a}_j \ \text{mod} \ q
	\end{equation}
\end{enumerate}

Donde $\phi$ es un valor tal que $\phi^2=\omega$, con $\omega$ una raíz enésima de la unidad en \(\mathbb{Z}_q\) y \(n^{-1}\) es la inversa multiplicativa de \(n\) en \(\mathbb{Z}_q\).
\newline

A continuación, se presenta un ejemplo extraído de \cite{cryptoeprint:2024/585} para ilustrar su funcionamiento. Sea el polinomio \(G(x)=5+6x+7x^2+8x^3\), cuyo vector de coeficientes es \(g=[5,6,7,8]\). Trabajando en el anillo \(\mathbb{Z}_{7681}\), y tomando $\phi=1925$, se puede calcular la transformada \gls{ntt} \(\hat{g}\). Aplicando luego la transformada inversa, es posible recuperar el vector original \(g\).
\begin{equation}
	\hat{g}=\begin{bmatrix}
		\phi^0 & \phi^1 & \phi^2 & \phi^3\\
		\phi^0 & \phi^3 & \phi^6 & \phi^1\\
		\phi^0 & \phi^5 & \phi^2 & \phi^7\\
		\phi^0 & \phi^7 & \phi^6 & \phi^5\\
	\end{bmatrix} \cdot \begin{bmatrix}
	5\\
	6\\
	7\\
	8
	\end{bmatrix}=\begin{bmatrix}
	1 & 1925 & 3383 & 6468\\
	1 & 6468 & 4298 & 1925\\
	1 & 5756 & 3383 & 1213\\
	1 & 1213 & 4298 & 5756\\
	\end{bmatrix} \cdot \begin{bmatrix}
	5\\
	6\\
	7\\
	8 \end{bmatrix}= \begin{bmatrix}
	2489\\
	7489\\
	6478\\
	6607 \end{bmatrix}
\end{equation}

Aplicando la transformada inversa, donde la inversa de $\phi=1925$ en \(\mathbb{Z}_{7681}\) es $\phi^{-1}=1213$ y el inverso del orden del polinomio \(n=4\) es \(n^{-1}=5761\):
\begin{equation}
	g=n^{-1}\begin{bmatrix}
		\phi^0 & \phi^0 & \phi^0 & \phi^0\\
		\phi^{-1} & \phi^{-3} & \phi^{-5} & \phi^{-7}\\
		\phi^{-2} & \phi^{-6} & \phi^{-2} & \phi^{-6}\\
		\phi^{-3} & \phi^{-1} & \phi^{-7} & \phi^{-5}\\
	\end{bmatrix} \cdot \hat{g}=5761\begin{bmatrix}
	1 & 1 & 1 & 1\\
	1213& 5756& 6468& 1925\\
	4298& 3383& 4298& 3383\\
	5756& 1213& 1925& 6468\\
	\end{bmatrix} \cdot \begin{bmatrix}
	2489\\
	7489\\
	6478\\
	6607 \end{bmatrix}
\end{equation}

Con estas transformadas definidas se define el producto o convolución negativa en el anillo \(	R_q \coloneqq \dfrac{\mathbb{Z}_q[X]}{X^n + 1}\) entre dos polinomios \(g\) y \(h\) como:
\begin{equation}
	g\cdot h= \text{NTT}^{-1}(\text{NTT}(g)\circ\text{NTT}(h))
\end{equation}

Donde la operación \(\circ\) denota la multiplicación elemento a elemento (o punto a punto) entre los vectores en ${Z}_q[X]$.
\newpage

Ahora, queda demostrar que el tiempo de ejecución de la \gls{ntt} es de \(\mathcal{O}(n \log(n))\). Para lograr este cometido, se aprovechan dos propiedades que cumplen estas raíces calculadas:
\begin{enumerate}
	\item Peridicidad: 
	\begin{equation}
		\phi^{k+2n}=\phi^k
	\end{equation}
	\item Simetría:
	\begin{equation}
		\phi^{k+n}=\phi^{-k}
	\end{equation}
\end{enumerate}

Con estas propiedades, se puede implementar el algoritmo de Cooley-Tukey \cite{CooleyALG} que consiste en ir descomponiendo el problema en mitades de manera recursiva para reducir al máximo la cantidad de cálculos realizados. Partiendo de la transformada directa y desarrollando:
\begin{equation}
	\hat{a}_j=\sum_{i=0}^{n-1} \phi^{i\left(2j+1\right)} a_j \ \text{mod} \ q = \sum_{i=0}^{n/2-1} \phi^{4ij+2i} a_{2i} + \phi^{2j+1} \sum_{i=0}^{n/2-1} \phi^{4ij+2i} a_{2i+1} \ \text{mod} \ q
\end{equation}

Si se sustituye \(A_j=\sum_{i=0}^{n/2-1} \phi^{4ij+2i} a_{2i}\) y \(B_j=\sum_{i=0}^{n/2-1} \phi^{4ij+2i} a_{2i+1}\). Ahora aplicando simetría en \(\phi\):
\begin{equation}
	\hat{a}_j=A_j+\phi^{4ij+2i}B_j  \ \text{mod} \ q
\end{equation}
\begin{equation}
	\hat{a}_{j+n/2}=A_j-\phi^{4ij+2i}B_j  \ \text{mod} \ q
\end{equation}

Donde las matrices \(A_j\) y \(B_j\)pueden obtenerse como el resultado de aplicar la \gls{ntt} sobre la mitad de los puntos, gracias a la estructura recursiva del algoritmo. Esto implica que, si \(n\) es una potencia de \(2\), el proceso puede repetirse recursivamente sobre subproblemas de tamaño cada vez menor, hasta alcanzar el caso base. 
\newline

De manera similar, se puede mostrar esta propiedad para la transformada inversa. Por tanto, se demuestra que este algoritmo tiene complejidad \(\mathcal{O}(n \log(n))\).
\newline

En el caso concreto de kyber \cite{kyber-spec-2021}, la Transformada (\gls{ntt}) de un polinomio en el anillo \(R_q\) se representa como un vector de 128 polinomios de grado 1.
\newline

Sean las 256 raíces enésimas de la unidad \(\left\{\xi, \xi^3, \dots, \xi^{255}\right\}\), con \(\xi = 17\) como primera raíz primitiva. Entonces, se cumple que:
\begin{equation}
	X^{256}+1=\prod_{i=0}^{127}\left(X^2- \xi^{2i+1}\right)
\end{equation}

Aplicando la \gls{ntt} propuesta en \cite{kyber-spec-2021} a un polinomio \(f\in R_q\) se obtiene:
\begin{equation}
	NTT(f)=\hat{f}=\left(\hat{f}_0+\hat{f}_1X,\hat{f}_2+\hat{f}_3X,...,\hat{f}_{254}+\hat{f}_{255}X\right)
\end{equation}

Con
\begin{equation}
	\begin{array}{l}
		\hat{f}_{2i}=\sum\limits_{j=0}^{127} f_{2j}\xi^{\left(2i+1\right)j}\\
		\hat{f}_{2i+1}=\sum\limits_{j=0}^{127} f_{2j+1}\xi^{\left(2i+1\right)j}
	\end{array}
\end{equation}

Donde mediante la transformada directa \gls{ntt} e inversa \gls{ntt}$^{-1}$ se puede realizar el producto de \(f\),\(g\in R_q\) de manera eficiente de la siguiente manera: 
\begin{equation}
	h=f\cdot g = \text{NTT}^{-1}\left[\text{NTT}(f) \circ \text{NTT}(g)\right]
\end{equation}

Siendo \(\hat{h}=\hat{f}\circ\hat{g}=\text{NTT}(f)\circ \text{NTT}(g)\) la multiplicación base definida como:
\begin{equation}
	\hat{h}_{2i}+\hat{h}_{2i+1}X=(\hat{f}_{2i}+\hat{f}_{2i+1})\left(\hat{g}_{2i}+\hat{g}_{2i+1}\right) \ \text{mod} \left(X^2-\xi^{2i+1}\right)
\end{equation}


En la figura \ref{fig:NTTFigure} se puede ver una representación gráfica de como funciona este proceso.
\begin{figure}[H]
	\centering
		\begin{circuitikz}
			\tikzstyle{every node}=[font=\LARGE]
			\draw  (5.25,13.75) rectangle (7.75,12.25);
			\draw  (13.5,13.75) rectangle (16,12.25);
			\node [font=\LARGE] at (6.5,13) {f $\cdot$ g};
			\node [font=\LARGE] at (14.75,13) {h};
			\draw [ color={rgb,255:red,255; green,40; blue,40}, ->, >=Stealth] (7.75,13) -- (13.5,13);
			\node [font=\LARGE, color={rgb,255:red,255; green,0; blue,0}] at (10.5,13.5) {$\mathcal{O}(n^2)$};
			\node [font=\normalsize] at (10.5,12.5) {Multiplicación directa};
			\draw  (5.25,9.5) rectangle (7.75,8);
			\draw  (13.5,9.5) rectangle (16,8);
			\draw [ color={rgb,255:red,0; green,64; blue,255}, ->, >=Stealth] (6.5,12.25) -- (6.5,9.5);
			\draw [ color={rgb,255:red,0; green,64; blue,255}, ->, >=Stealth] (7.75,8.75) -- (13.5,8.75);
			\draw [ color={rgb,255:red,0; green,64; blue,255}, ->, >=Stealth] (14.75,9.5) -- (14.75,12.25);
			\node [font=\LARGE] at (5.75,10.75) {NTT};
			\node [font=\LARGE] at (15.75,10.75) {NTT$^{-1}$};
			\node [font=\LARGE, color={rgb,255:red,0; green,64; blue,255}] at (10.5,9.25) {$\mathcal{O}(n\log(n))$};
			\node [font=\normalsize] at (10.75,8.25) {Multiplicación mediante NTT};
			\node [font=\LARGE] at (6.5,8.75) {$\hat{f}$ $\circ$ $\hat{g}$};
			\node [font=\LARGE] at (14.75,8.75) {$\hat{h}$};
		\end{circuitikz}
	\label{fig:NTTFigure}
	\caption{Representación del cálculo de un producto de polinomios mediante \gls{ntt} en comparación con los algoritmos clásicos.}
\end{figure}


\subsubsection{Aprendizaje Con Errores o Learning With Errors (\gls{lwe})}
\cite{LWElearning}
\subsection{Transformadas Fujisaki-Okamoto}
\cite{Fujisaki1999}
\subsubsection{Algoritmos principales}
\newpage
\subsection{SABER}
Para \cite{saber-spec-2020}
\subsection{Hamming Quasi-Cyclic (HQC)}
Para  \cite{hqc-spec-2022}
\subsection{Bit Flipping Key Encapsulation (Bike)}
Para \cite{bike-spec-2022}
\newpage
\section{Fundamentos de seguridad de los algoritmos}
\subsection{CRYSTALS-Kyber }
Para \cite{kyber-spec-2021}
\subsection{SABER}
Para \cite{saber-spec-2020}
\subsection{Hamming Quasi-Cyclic (HQC)}
Para  \cite{hqc-spec-2022}
\subsection{Bit Flipping Key Encapsulation (Bike)}
Para \cite{bike-spec-2022}