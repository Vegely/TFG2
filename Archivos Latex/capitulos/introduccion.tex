\chapter{Introducción}

\section{Motivación del proyecto}
En el ámbito de las comunicaciones industriales, la seguridad en el intercambio de información es un aspecto crítico, especialmente ante el avance de la computación cuántica y su impacto en los algoritmos criptográficos actuales. Este trabajo se enfoca en el análisis, implementación y evaluación de algoritmos de cifrado asimétrico post-cuántico en sistemas embebidos, con el objetivo de garantizar la seguridad de los protocolos de comunicación en un entorno industrial. Se realiza un estudio detallado de los fundamentos matemáticos de los algoritmos asimétricos clásicos y su vulnerabilidad frente al algoritmo de Shor, así como de los nuevos esquemas criptográficos diseñados para resistir ataques cuánticos.
\newline

El cifrado asimétrico es esencial para establecer claves seguras en protocolos de intercambio de claves, permitiendo así la utilización de cifrado simétrico en las comunicaciones industriales. Esto es particularmente relevante en sistemas de control en línea, donde el cifrado simétrico debe operar dentro del bucle de control en tiempo real, garantizando tanto seguridad como eficiencia.
\newline

Para ello, en este trabajo se comparan diferentes candidatos propuestos en el estándar NIST para evaluar su rendimiento en términos de velocidad, consumo de recursos y nivel de seguridad. 
\section{Objetivos}
Para realizar el proyecto, se proponen los siguientes objetivos:
\begin{itemize}
\item Analizar el algoritmo de Shor y su impacto en la seguridad del cifrado asimétrico clásico.
\item Estudiar los fundamentos matemáticos de los métodos de cifrado asimétrico modernos.
\item Implementar un sistema de comunicación entre un PC y un microcontrolador para el intercambio de claves seguras.
\item  Desarrollar e integrar los siguientes algoritmos de cifrado asimétrico postcuántico en un microcontrolador y PC:
\begin{itemize}
	\item Kyber
	\item Saber
	\item HQC
\end{itemize}
\item Comparar el rendimiento de los algoritmos de cifrado postcuántico evaluando velocidad, consumo de recursos y seguridad.
\item Estudiar la necesidad de implementar estos algoritmos en FPGAs en caso de que en el microcontrolador el rendimiento no fuera aceptable.
\item Implementar y diseñar un sistema de intercambio de claves que permita la escalabilidad de la solución.
\end{itemize}


\section{Herramientas utilizadas}
\subsection{LaTex} 
Se ha preferido el uso de \LaTeX\ debido a la facilidad que ofrece para el maquetado de textos, superando a otras herramientas de elaboración de documentos. Además, \LaTeX\ permite crear figuras vectorizadas, representar correctamente ecuaciones y ubicar adecuadamente figuras, tablas y bibliografía.
\subsection{TikzMaker} 
Esta herramienta \cite{tikzmaker} permite crear figuras vectorizadas de \LaTeX\ mediante el paquete de circuitikz. Su principal ventaja radica en la interfaz gráfica que proporciona y en la facilidad para elaborar figuras.

\subsection{C y C++}

\subsection{Microprocesador CY8CPROTO-063-BLE }
se usa el \cite{CY8CPROTO063BLE}
\section{Estructura del documento}
A continuación y para facilitar la lectura del documento, se detalla el contenido de cada capítulo:

\begin{itemize}
	\item En el capítulo 1 se realiza una introducción.
	\item En el capítulo 2 se estudia trabajos realizados con relación al tema.
	\item En el capítulo 3 se desarrollan los fundamentos matemáticos del proyecto.
	\item En el capítulo 4 se describe la implementación de los algoritmos.
	\item En el capítulo 5 se presentan y analizan los resultados obtenidos en el capítulo anterior como consecuencia de ejecutar el software realizado. 
	\item En el capítulo 6  se realiza una conclusión.
\end{itemize}