\newacronym{ntt}{NTT}{Transformada Teórica de Números}
\newacronym{dft}{DFT}{Transformada Discreta de Fourier}
\newacronym{lwe}{LWE}{Aprendizaje Con Errores}
\newacronym{lwr}{LWR}{Aprendizaje Con Redondeo}
\newacronym{rlwe}{R-LWE}{Aprendizaje Con Errores en Anillos}
\newacronym{mlwe}{M-LWE}{Aprendizaje Con Errores Modular}
\newacronym{mlwr}{Mod-LWR}{Aprendizaje Con Redondeo Modular}
\newacronym{cca2}{IND-CCA2}{Indistinguibilidad bajo ataque de texto cifrado adaptable}
\newacronym{ecc}{ECC}{Criptografía en Curvas Elípticas}
\newacronym{rsa}{RSA}{Rivest-Shamir-Adleman}
\newacronym{tfo}{TFO}{Transformadas Fujisaki-Okamoto}
\newacronym{kem}{KEM}{Mecanismo de intercambio de claves}
\newacronym{nist}{NIST}{Instituto Nacional de Seguridad e Información}
\newacronym{hqc}{HQC}{Hamming Quasi-Cyclic}
\newacronym{sha}{SHA}{Algoritmo Seguro de Hashing }
\newacronym{fips}{FIPS}{Estandar Federal de Procesamiento de la Información}
\newacronym{qc}{QC}{Cuasi-cíclico}
\newacronym{ind}{IND}{Indistinguibilidad del cifrado}
\newacronym{nm}{NM}{No maleabilidad del cifrado}
\newacronym{cpa}{CPA}{Ataque de texto plano}
\newacronym{cca1}{CCA1}{Ataque de texto cifrado no adaptable}
\newacronym{cca2.}{CCA2}{Ataque de texto cifrado adaptable}
\newacronym{pa}{PA}{Ataque de texto cifrado adaptable}
\chapter{Glosario}
\section*{Abreviaturas y siglas}
\addcontentsline{toc}{section}{Abreviaturas y siglas}
\normalsize

\renewcommand*{\glossarysection}[2][]{}
\setglossarystyle{super}
\printglossary[type=\acronymtype]