% Plantilla realizada por Alberto Brunete (UPM).

%parametros de tipo libro
\documentclass[10pt,a4paper]{book}

%idioma español y acentos
\usepackage[spanish, es-tabla]{babel}
%\usepackage[latin1]{inputenc}
%\usepackage[utf8]{inputenc}

%algunos sÌmbolos matem·ticos y paquete para usar subim·genes
\usepackage{amsfonts}
\usepackage{graphicx}
\usepackage{subfigure}
\usepackage{listings}
\usepackage{appendix}
%M·rgenes
\usepackage[left=3cm,top=3cm,right=3cm,bottom=3cm]{geometry}

%
\usepackage{xcolor}
\usepackage{multicol}
\usepackage{pgfplots}
\pgfplotsset{compat=newest}
\usepackage{float} 
\usepackage{circuitikz}
\usepackage{tikz}
\usetikzlibrary{babel}
\usepackage{amsmath}
\usepackage{amssymb}
%para generar índice con hipervínculos
\usepackage{hyperref}
\usepackage{mathtools}
\usepackage{siunitx}
\usepackage{esint}
\usepackage{adjustbox}
\usepackage{algorithm}
\usepackage{algpseudocode}
\usepackage{amsthm}
\usetikzlibrary{tikzmark,calc}
\usepackage[acronym, nonumberlist]{glossaries}
\floatname{algorithm}{Algoritmo}
%parametros del documento (sus propiedades)
\hypersetup{
    pdftitle={Bogurad Barañski Barañska - TFG - 2025},
    pdfsubject={TFG - año},
    pdfauthor={Bogurad Barañski Barañska},
    pdfkeywords={encriptación} {comunicaciones} {lazo de control},
    colorlinks,
    citecolor=black,
    filecolor=black,
    linkcolor=black,
    urlcolor=black,
}


\definecolor{codegreen}{rgb}{0,0.6,0}
\definecolor{codegray}{rgb}{0.5,0.5,0.5}
\definecolor{codepurple}{rgb}{0.58,0,0.82}
\definecolor{backcolour}{rgb}{0.95,0.95,0.92}
\definecolor{codeblue}{rgb}{0.0, 0.2, 0.6}


\lstdefinestyle{mystyle}{
	backgroundcolor=\color{backcolour},   
	commentstyle=\color{codegreen},
	keywordstyle=\color{codeblue}\bfseries,
	numberstyle=\tiny\color{codegray},
	stringstyle=\color{codepurple},
	basicstyle=\ttfamily\footnotesize, % Fuente monoespaciada
	breakatwhitespace=false,         
	breaklines=true,                 
	captionpos=b,                    % <--- ESTO PONE EL CAPTION ABAJO
	keepspaces=true,                 
	numbers=left,                    % Números de línea a la izquierda
	numbersep=5pt,                  
	showspaces=false,                
	showstringspaces=false,
	showtabs=false,                  
	tabsize=2,
	frame=single,                    % Un marco alrededor del código
	rulecolor=\color{black!30},      % Color del marco suave
}

\lstset{style=mystyle}
\newcommand{\esquote}[1]{``#1''}

%indice de gloarios 
\theoremstyle{definition}
\newtheorem{theorem}{Teorema}
\renewcommand{\proofname}{Demostración}

% profundidad de numeracion
\setcounter{secnumdepth}{3}
% profundidad de tabla de contenidos
\setcounter{tocdepth}{4}


%empieza el documento
\newglossary[sym]{symbols}{sym}{sbl}{Símbolos}
\makeglossaries

\begin{document}  
		
	%elementos antes del trabajo en sÌ se meten dentro de frontmatter
	\frontmatter
	
	%cada incluye referencia a un archivo de tipo .tex
	\begin{titlepage}
\begin{center}

%forma de introducir imágenes. el \\[0.5 cm] de final de línea introduce un salto de ese tamaño.
%width=1\textwidth indica el tamaño de la imágen (valores entre 0-1). 
 \includegraphics[width=1\textwidth]{figuras/cabecera.png}  \\[0.5 cm]

\LARGE UNIVERSIDAD POLITÉCNICA DE MADRID \\ [1 cm]

\LARGE ESCUELA TÉCNICA SUPERIOR DE INGENIERÍA Y DISEÑO INDUSTRIAL \\ [1 cm]

\LARGE Grado en Ingeniería Electrónica Industrial y Automática\\ [1 cm]

\LARGE \textbf{TRABAJO FIN DE GRADO}\\[1 cm]

\Huge \textsc{Estudio comparativo de diferentes algoritmos de encriptación para comunicaciones industriales}\\[1 cm]

\LARGE Bogurad Barañski Barañska \\[1.5 cm]

%flushleft alinea a la izquierda el texto

\begin{multicols}{2} 
\begin{flushleft} \large	
\emph{Cotutor:} Basil Mohammed Al-Hadithi\\
\emph{Departamento:} ingeniería eléctrica, electrónica, automática y física aplicada.
\end{flushleft}

\begin{flushright} \large	
\emph{Tutor:}  Roberto Gonzalez Herranz\\
\emph{Departamento:} ingeniería eléctrica, electrónica, automática y física aplicada.
\end{flushright}

\end{multicols} 

%rellena de blanco el resto de la página para escribir abajo del todo
\vfill

% Bottom of the page
{\large Madrid, Septiembre, 2025}

%SE ponen al final firmas.tex
%\end{center}
%\end{titlepage}


\cleardoublepage 
	
	\include{capitulos/firmas}
	
	%Licencia opcional
	\include{capitulos/licencia}
	
	\cleardoublepage

\begin{flushleft} \large
\textbf{Título:} Estudio comparativo de diferentes algoritmos de encriptación para comunicaciones industriales \\
\textbf{Autor:} Bogurad Barañski Barañska\\
\textbf{Tutor:} Roberto Gonzalez Herranz \\ 
\textbf{Cotutor:} Basil Mohammed Al-Hadithi \\ [1 cm]
\end{flushleft} 

\begin{center} \LARGE
EL TRIBUNAL \\ [1 cm]
\end{center}

\begin{flushleft} \LARGE
Presidente: \\ [1 cm]
Vocal: \\ [1 cm]
Secretario: \\ [1.5 cm]
\end{flushleft}

\large
Realizado el acto de defensa y lectura del Trabajo Fin de Grado el día ....... de ....................   de ... en .........., en la Escuela Técnica Superior de Ingeniería y Diseño Industrial de la Universidad Politécnica de Madrid, acuerda otorgarle la CALIFICACIÓN de: \par

\begin{center}
 \large VOCAL \\ [2.2 cm]
\end{center}

\begin{minipage}{0.5\textwidth}
 \begin{flushleft}
 \large SECRETARIO
\end{flushleft}
\end{minipage}
\begin{minipage}{0.5\textwidth}
\begin{flushright}
 \large PRESIDENTE
\end{flushright} 
\end{minipage}
	\include{capitulos/agradecimientos}
	%chapter introduce un nuevo capítulo
\chapter{Resumen}

Este proyecto aborda el desafío crítico que supone la llegada de la computación cuántica para la seguridad de las comunicaciones industriales actuales. Ante la vulnerabilidad de los estándares de cifrado asimétrico tradicionales, se presenta un estudio comparativo y la implementación práctica de tres algoritmos de criptografía postcuántica (PQC) seleccionados en el proceso de estandarización del NIST: CRYSTALS-Kyber, Saber y HQC.
\newline

El desarrollo se ha realizado sobre un sistema embebido de recursos limitados, utilizando un microcontrolador PSOC 6 BLE (ARM Cortex-M4), para evaluar la viabilidad de estos algoritmos en entornos de operación real. Se ha diseñado una arquitectura Cliente-Servidor con comunicación vía puerto serie para realizar pruebas de intercambio de claves (KEM).
\newline

Los resultados experimentales analizan métricas clave como el tiempo de cómputo, el consumo de memoria y la entropía de los generadores de números aleatorios. El análisis concluye que, mientras algoritmos como Kyber y Saber ofrecen un rendimiento muy eficiente y apto para la industria, alternativas como HQC, aunque seguras, presentan mayores exigencias de memoria que pueden limitar su uso en dispositivos de gama baja.

\section*{Palabras clave:} Criptografía Postcuántica, Ciberseguridad Industrial, Sistemas Embebidos, PSOC 6, NIST, Kyber, Saber, HQC.



\chapter{Abstract}

This project addresses the critical challenge posed by the advent of quantum computing to the security of current industrial communications. In light of the vulnerability of traditional asymmetric encryption standards, this work presents a comparative study and practical implementation of three Post-Quantum Cryptography (PQC) algorithms selected in the NIST standardization process: CRYSTALS-Kyber, Saber, and HQC.
\newline

The development was carried out on a resource-constrained embedded system, utilizing a PSOC 6 BLE microcontroller (ARM Cortex-M4), to evaluate the viability of these algorithms in real-world operating environments. A Client-Server architecture with serial port communication was designed to conduct Key Encapsulation Mechanism (KEM) tests.
\newline

Experimental results analyze key metrics such as computation time, memory consumption, and the entropy of random number generators. The analysis concludes that, while algorithms such as Kyber and Saber offer highly efficient performance suitable for industrial applications, alternatives like HQC, although secure, present higher memory requirements that may limit their use in low-end devices.

\section*{Keywords:} Post-Quantum Cryptography, Industrial Cybersecurity, Embedded Systems, PSOC 6, NIST, Kyber, Saber, HQC.
	
	%genera índice
	\tableofcontents
	\addcontentsline{toc}{chapter}{Índice}
	
	%Índice de figuras.
	\listoffigures
	
	%Índice de tablas.
	\listoftables
	
	\newacronym{ntt}{NTT}{Transformada Teórica de Números}
\newacronym{dft}{DFT}{Transformada Discreta de Fourier}
\newacronym{lwe}{LWE}{Aprendizaje Con Errores}
\newacronym{lwr}{LWR}{Aprendizaje Con Redondeo}
\newacronym{rlwe}{R-LWE}{Aprendizaje Con Errores en Anillos}
\newacronym{mlwe}{M-LWE}{Aprendizaje Con Errores Modular}
\newacronym{mlwr}{Mod-LWR}{Aprendizaje Con Redondeo Modular}
\newacronym{cca2}{IND-CCA2}{Indistinguibilidad bajo ataque de texto cifrado adaptable}
\newacronym{ecc}{ECC}{Criptografía en Curvas Elípticas}
\newacronym{rsa}{RSA}{Rivest-Shamir-Adleman}
\newacronym{tfo}{TFO}{Transformadas Fujisaki-Okamoto}
\newacronym{kem}{KEM}{Mecanismo de intercambio de claves}
\newacronym{nist}{NIST}{Instituto Nacional de Seguridad e Información}
\newacronym{hqc}{HQC}{Hamming Quasi-Cyclic}
\newacronym{sha}{SHA}{Algoritmo Seguro de Hashing }
\newacronym{fips}{FIPS}{Estandar Federal de Procesamiento de la Información}
\newacronym{qc}{QC}{Cuasi-cíclico}
\newacronym{ind}{IND}{Indistinguibilidad del cifrado}
\newacronym{nm}{NM}{No maleabilidad del cifrado}
\newacronym{cpa}{CPA}{Ataque de texto plano}
\newacronym{cca1}{CCA1}{Ataque de texto cifrado no adaptable}
\newacronym{cca22}{CCA2}{Ataque de texto cifrado adaptable}
\newacronym{pa}{PA}{Consciencia de texto plano}
\newacronym{rom}{ROM}{Modelo de oráculo aleatorio}
\newacronym{qrom}{QROM}{Modelo de oráculo cuántico aleatorio}
\chapter{Abreviaturas y siglas}
\normalsize
\renewcommand*{\glossarysection}[2][]{}
\printglossary[type=\acronymtype]
	%Empieza la parte descriptiva del trabajo
	\mainmatter
	
	\chapter{Introducción}

\section{Motivación del proyecto}
En el ámbito de las comunicaciones industriales, la seguridad en el intercambio de información es un aspecto crítico, especialmente ante el avance de la computación cuántica y su impacto en los algoritmos criptográficos actuales. Este trabajo se enfoca en el análisis, implementación y evaluación de algoritmos de cifrado asimétrico post-cuántico en sistemas embebidos, con el objetivo de garantizar la seguridad de los protocolos de comunicación en un entorno industrial. Se realiza un estudio detallado de los fundamentos matemáticos de los algoritmos asimétricos clásicos y su vulnerabilidad frente al algoritmo de Shor, así como de los nuevos esquemas criptográficos diseñados para resistir ataques cuánticos.
\newline

El cifrado asimétrico es esencial para establecer claves seguras en protocolos de intercambio de claves, permitiendo así la utilización de cifrado simétrico en las comunicaciones industriales. Esto es particularmente relevante en sistemas de control en línea, donde el cifrado simétrico debe operar dentro del bucle de control en tiempo real, garantizando tanto seguridad como eficiencia.
\newline

Para ello, en este trabajo se comparan diferentes candidatos propuestos en el estándar NIST para evaluar su rendimiento en términos de velocidad, consumo de recursos y nivel de seguridad. 
\section{Objetivos}
Para realizar el proyecto, se proponen los siguientes objetivos:
\begin{itemize}
\item Analizar el algoritmo de Shor y su impacto en la seguridad del cifrado asimétrico clásico.
\item Estudiar los fundamentos matemáticos de los métodos de cifrado asimétrico modernos.
\item Implementar un sistema de comunicación entre un PC y un microcontrolador para el intercambio de claves seguras.
\item  Desarrollar e integrar los siguientes algoritmos de cifrado asimétrico postcuántico en un microcontrolador y PC:
\begin{itemize}
	\item Kyber
	\item Saber
	\item HQC
\end{itemize}
\item Comparar el rendimiento de los algoritmos de cifrado postcuántico evaluando velocidad, consumo de recursos y seguridad.
\item Estudiar la necesidad de implementar estos algoritmos en FPGAs en caso de que en el microcontrolador el rendimiento no fuera aceptable.
\item Implementar y diseñar un sistema de intercambio de claves que permita la escalabilidad de la solución.
\end{itemize}


\section{Herramientas utilizadas}
\subsection{LaTex} 
Se ha preferido el uso de \LaTeX\ debido a la facilidad que ofrece para el maquetado de textos, superando a otras herramientas de elaboración de documentos. Además, \LaTeX\ permite crear figuras vectorizadas, representar correctamente ecuaciones y ubicar adecuadamente figuras, tablas y bibliografía.
\subsection{TikzMaker} 
Esta herramienta \cite{tikzmaker} permite crear figuras vectorizadas de \LaTeX\ mediante el paquete de circuitikz. Su principal ventaja radica en la interfaz gráfica que proporciona y en la facilidad para elaborar figuras.

\subsection{C y C++}

\subsection{Microprocesador CY8CPROTO-063-BLE }
se usa el \cite{CY8CPROTO063BLE}
\section{Estructura del documento}
A continuación y para facilitar la lectura del documento, se detalla el contenido de cada capítulo:

\begin{itemize}
	\item En el capítulo 1 se realiza una introducción.
	\item En el capítulo 2 se estudia trabajos realizados con relación al tema.
	\item En el capítulo 3 se desarrollan los fundamentos matemáticos del proyecto.
	\item En el capítulo 4 se describe la implementación de los algoritmos.
	\item En el capítulo 5 se presentan y analizan los resultados obtenidos en el capítulo anterior como consecuencia de ejecutar el software realizado. 
	\item En el capítulo 6  se realiza una conclusión.
\end{itemize}
	
	%partes finales del trabajo: conclusiones, bibliografia y anexos
	
	\chapter{Estado del arte}
En este capítulo se discute la literatura general existente respecto a sistemas de cifrado. Los detalles de literatura más relevantes respecto a los algoritmos concretos utilizados en este trabajo fin de grado se encuentran en el capítulo de fundamentos.
\section{Introducción}
Una vez establecida la motivación del proyecto y la necesidad crítica de asegurar los sistemas de control industrial frente a futuras amenazas, es imprescindible contextualizar el estado actual de la tecnología criptográfica. Este capítulo ofrece una visión integral sobre la evolución de los estándares de seguridad, analizando el paradigma de la criptografía de clave pública actual y el punto de inflexión que supone la aparición de la computación cuántica.
\newline

En el panorama actual, la protección de datos se rige por marcos regulatorios y técnicos complementarios. Mientras que en Estados Unidos el Instituto Nacional de Estándares y Tecnología (\acrshort{nist}) define especificaciones técnicas concretas, en Europa el Reglamento General de Protección de Datos (GDPR) establece la obligatoriedad legal de garantizar la seguridad de la información. Ambos enfoques convergen en la recomendación de estándares robustos \cite{A_COMPARATIVE_REVIEW_OF_DATA_ENCRYPTION_METHODS_IN_THE_USA_AND_EUROPE}, consolidando el uso de algoritmos simétricos como \acrfull{aes} para el cifrado de datos.
\newline

No obstante, a diferencia de la estabilidad del cifrado simétrico, los mecanismos de establecimiento de claves atraviesan una transformación profunda. Este proceso de estandarización, motivado por la amenaza cuántica, ha sido un esfuerzo global liderado por el \acrshort{nist}, iniciándose formalmente con las rondas de selección en 2017-2019 y culminando sus primeras fases de estandarización recientemente entre 2024 y 2025 \cite{NIST_IR_8240_2019}\cite{NIST_IR_8309_2020}\cite{NIST_IR_8413_2022}\cite{NIST_IR_8545_2025}.
\newline

En este capítulo, en primer lugar, se revisan brevemente los criptosistemas clásicos que han dominado la seguridad informática en las últimas décadas, identificando las limitaciones que presentan ante el nuevo escenario tecnológico, específicamente debido a la vulnerabilidad frente al algoritmo de Shor y el incremento en la capacidad de cómputo cuántico.
\newline

Finalmente, se detalla el mencionado proceso de estandarización de criptografía postcuántica. Se examinan los candidatos finalistas y alternativos, cuyo estudio comparativo fundamenta la selección de los algoritmos que serán descritos matemáticamente en el capítulo de Fundamentos Generales y posteriormente implementados en la arquitectura propuesta.
\newpage

\section{Tipos de cifrado}
Como se introdujo en la Sección \ref{sec:moti}, la criptografía moderna se divide en dos paradigmas fundamentales: simétrica y asimétrica.
\newline

La criptografía simétrica se basa en el principio de secreto compartido, donde se utiliza una única clave tanto para el proceso de cifrado como para el descifrado. Matemáticamente, el emisor transforma el texto plano utilizando la clave y el receptor realiza el proceso inverso transformando el texto cifrado en texto plano. Sin embargo, este modelo presenta un desafío inherente conocido como el problema de distribución de claves. Donde para que el sistema sea seguro, ambas partes deben haber acordado la clave previamente a través de un canal seguro. En un entorno de red abierto, transmitir esta clave sin protección comprometería toda la comunicación.
\newline

Para mitigar esta vulnerabilidad surge la criptografía asimétrica (o de clave pública). Este enfoque soluciona el problema del intercambio de secretos mediante el uso de un par de claves matemáticamente vinculadas: una clave pública para cifrar, que puede ser distribuida abiertamente, y una clave privada para descifrar, que permanece bajo custodia exclusiva del receptor.
\newline

Ante esta disyuntiva, resulta pertinente cuestionar la vigencia del cifrado simétrico si la criptografía asimétrica ya resuelve la problemática distribución de claves. La justificación fundamental reside en la eficiencia computacional. Mientras que los criptosistemas asimétricos dependen de operaciones matemáticas intensivas que introducen un elevado sobrecoste, los algoritmos simétricos están diseñados estructuralmente para el procesamiento de datos a alta velocidad. La evidencia empírica y los benchmarks de la industria confirman esta disparidad: el cifrado simétrico es entre \textbf{100 y 4.000} veces más rápido que su contraparte asimétrica, una diferencia de magnitud que varía según el algoritmo específico y el volumen de datos a procesar \cite{CryptoPP_Benchmarks} \cite{Comparative_Analysis_of_Energy_Costs_of_Asymmetric_vs_Symmetric_Encryption-Based_Security_Applications}.
\newline

Por consiguiente, la arquitectura estándar actual es híbrida: se utiliza criptografía asimétrica exclusivamente para establecer la clave, y criptografía simétrica para el cifrado masivo de los datos \cite{An_Overview_and_Analysis_of_Hybrid_Encryption_The_Combination_of_Symmetric_Encryption_and_Asymmetric_Encryption}\cite{2022-414}. 

\section{Algoritmos de cifrado clásicos. Algoritmo de Shor}
En el ámbito del intercambio de claves clásico se han utilizado predominantemente dos algoritmos: \acrfull{rsa} y \acrfull{ecc}. Sin embargo, ambos se han demostrado vulnerables ante la computación cuántica como consecuencia del algoritmo de Shor. Pese a ello, continúan utilizándose en numerosas aplicaciones actuales, aunque la literatura reciente advierte sobre la necesidad de su obsolescencia \cite{ugwuishiwu2020overview}. El riesgo subyacente radica en que un atacante podría almacenar los mensajes cifrados hoy para descifrarlos en el futuro. Por tanto, la confidencialidad de la información crítica a largo plazo queda comprometida si depende exclusivamente de estas claves.
\newline

El algoritmo de Shor \cite{9508027v2} es un algoritmo eficiente para la factorización en números primos que utiliza las propiedades de la mecánica cuántica, específicamente el paralelismo cuántico y la interferencia, para resolver problemas intratables para la computación clásica en tiempo polinomial. Mientras que los mejores algoritmos clásicos para factorización y logaritmo discreto requieren tiempos superpolinomiales o exponenciales, el algoritmo de Shor ofrece una complejidad de $O(n^3)$. Su núcleo se basa en la Transformada Cuántica de Fourier (\acrshort{qft}), la cual permite encontrar el periodo de una función oscilatoria de manera exponencialmente más rápida que cualquier método conocido basado en física clásica.
\newline

\acrshort{rsa} es un algoritmo que fundamenta su seguridad en la dificultad computacional de factorizar grandes números enteros compuestos, típicamente el producto $N$ de dos números primos grandes $p$ y $q$. En el artículo \cite{RESERCHFINAL} y en estudios complementarios sobre la eficiencia de estos ataques \cite{bhatia2020efficient}, se demuestra que falla ante ataques cuánticos porque el algoritmo de Shor reduce el problema de la factorización a un problema de búsqueda de periodos. Utilizando la superposición cuántica, el algoritmo evalúa la función modular $f(x) = a^x \pmod N$ para múltiples valores simultáneamente y, mediante la \acrshort{qft}, extrae el periodo $r$ de dicha función. Una vez obtenido $r$, es computacionalmente sencillo para un ordenador clásico calcular los factores primos $p$ y $q$ mediante el máximo común divisor de $(a^{r/2} \pm 1, N)$, rompiendo así la clave privada.
\newline

\acrshort{ecc} es un algoritmo que basa su seguridad en la dificultad del Problema del Logaritmo Discreto en Curvas Elípticas sobre cuerpos finitos, permitiendo el uso de claves mucho más cortas que \acrshort{rsa} para un nivel de seguridad equivalente. Sin embargo, en el artículo \cite{0301141v2} se demuestra que falla ante ataques cuánticos porque una variante del algoritmo de Shor puede resolver el problema del logaritmo discreto en el grupo de puntos de la curva elíptica de manera eficiente. Es crucial destacar que \acrshort{ecc} resulta ser comparativamente más vulnerable a la computación cuántica que \acrshort{rsa} por bit de seguridad. Se estima que una clave de \acrshort{ecc} de 160 bits podría romperse utilizando aproximadamente 1000 qubits, mientras que factorizar un módulo \acrshort{rsa} de 1024 bits (que ofrece una seguridad clásica similar) requeriría alrededor de 2000 qubits. Esto implica que los ordenadores cuánticos más pequeños podrán comprometer sistemas \acrshort{ecc} antes de tener la capacidad de romper sistemas \acrshort{rsa} equivalentes.
\newline

Además, hoy en día se están haciendo rápidos avances, como se ve en las hojas de ruta hacia la computación cuántica tolerante a fallos propuesta por Microsoft Quantum \cite{microsoft2025roadmap}. Estos avances apuntan hacia la construcción de dispositivos con miles de qubits lógicos capaces de ejecutar algoritmos complejos como el de Shor, acelerando la llegada del momento en que la criptografía clásica quede obsoleta.


\section{Evolución de los algoritmos postcuánticos. Rondas de selección del NIST}
Ante la amenaza inminente que supone la computación cuántica para los estándares criptográficos actuales, el \acrshort{nist} inició en 2016 un proceso público de estandarización con el objetivo de seleccionar y evaluar algoritmos resistentes a ataques cuánticos. Este proceso no se planteó como una competición cerrada, sino como un esfuerzo colaborativo para identificar esquemas seguros, eficientes y flexibles para cifrado de clave pública, establecimiento de claves (\acrshort{kem}) y firma digital.
\newline

El proceso comenzó formalmente con la convocatoria de propuestas en diciembre de 2016. En la primera ronda, que se extendió hasta 2019, se recibieron un total de 82 propuestas, de las cuales 69 cumplieron los criterios mínimos de aceptación y requerimientos de presentación. Esta fase inicial se caracterizó por una gran diversidad de enfoques matemáticos, incluyendo retículas, códigos correctores de errores, isogenias y sistemas multivariables \cite{NIST_IR_8240_2019}.
\newline

Tras un periodo de análisis de seguridad y rendimiento, el \acrshort{nist} seleccionó 26 algoritmos para avanzar a la segunda ronda (2019-2020). Durante esta etapa, la comunidad criptográfica intensificó el criptoanálisis, descartando aquellos candidatos que presentaban vulnerabilidades o un rendimiento deficiente en comparación con sus competidores. De los candidatos supervivientes, se observó un predominio de las soluciones basadas en retículas estructuradas debido a su equilibrio entre tamaño de clave y velocidad de cómputo \cite{NIST_IR_8309_2020}.
\newline

La tercera ronda (2020-2022) marcó un hito decisivo en el proceso. El \acrshort{nist} identificó siete finalistas y ocho candidatos alternativos. En julio de 2022, se anunciaron los primeros algoritmos ganadores para ser estandarizados. Para el establecimiento de claves (\acrshort{kem}), se seleccionó el algoritmo CRYSTALS-Kyber, destacando por su eficiencia \cite{NIST_IR_8413_2022}.
\newline

Sin embargo, buscando la diversificación de fundamentos matemáticos para no depender exclusivamente de la seguridad de las retículas, el proceso continuó hacia una cuarta ronda (2022-2025). Esta fase se centró en evaluar \acrshort{kem}s alternativos basados en códigos y otras primitivas. Los candidatos evaluados incluyeron BIKE, Classic McEliece, \acrshort{hqc} y SIKE. Durante este periodo, se demostró la importancia de la diversificación cuando el candidato SIKE (basado en isogenias) fue roto criptográficamente. Finalmente, en marzo de 2025, el \acrshort{nist} anunció la selección de \acrshort{hqc} (basado en códigos) para su estandarización, complementando así a los estándares basados en retículas \cite{NIST_IR_8545_2025}.
\newline

Como resultado de este exhaustivo proceso de casi una década, en agosto de 2024 se publicó oficialmente el estándar FIPS 203, que especifica el Mecanismo de Encapsulamiento de Claves Basado en Retículas de Módulos (ML-KEM), derivado directamente de CRYSTALS-Kyber \cite{NISTFIPS203}.



\section{Distribución Cuántica de Claves}
En contraposición a la metodología del \acrshort{nist}, que se centra en la estandarización de algoritmos de Criptografía postcuántica basados en la complejidad computacional, la Distribución Cuántica de Claves (\acrshort{qkd}) propone un paradigma de seguridad basado en leyes físicas. Este paradigma se sustenta en el hecho de que el enfoque computacional del \acrshort{nist} presenta riesgos ante avances que demuestren la vulnerabilidad de estos algoritmos, tal como ya ocurrió en rondas anteriores con candidatos como SIKE, los cuales fueron rotos mediante nuevos ataques criptoanalíticos ejecutados en hardware clásico \cite{RubioGarcia:2023dgz}. 
\newline

Frente a este escenario de incertidumbre algorítmica, \acrshort{qkd} emerge como una alternativa que ofrece Seguridad Teórica de la Información, independiente de la capacidad computacional del adversario \cite{RubioGarcia:2023dgz}. No obstante, a pesar de sus garantías teóricas, su aplicación práctica en un entorno industrial se ve obstaculizada por barreras de implementación críticas. Una de las restricciones más severas es la exigencia de colocalización física, la cual dicta que la entidad de aplicación y el nodo \acrshort{qkd} deben estar directamente conectados para evitar la exposición de claves en enlaces clásicos, lo que obliga a utilizar hardware dedicado que resulta prohibitivo para la escalabilidad en redes industriales dispersas \cite{RubioGarcia:2023dgz} \cite{s11042-024-20535-x}. 
\newline 

A esta complejidad de infraestructura, que demanda canales ópticos duales y hardware específico, se suma un impacto negativo considerable en el rendimiento de la red. La evidencia experimental indica que la integración de \acrshort{qkd} en protocolos estándar introduce una sobrecarga de comunicación cercana al 117\%, derivada principalmente de la alta latencia en las interfaces de recuperación de claves, lo cual es incompatible con los estrictos requisitos de tiempo real necesarios para el control de procesos críticos \cite{RubioGarcia:2023dgz}. Asimismo, la gestión de recursos criptográficos presenta vulnerabilidades operativas significativas frente a ataques de agotamiento de claves. Dado que la tasa de generación de claves es un recurso finito que decae con la distancia, un atacante podría saturar el sistema mediante solicitudes masivas, obligando a implementar mecanismos de autenticación previa que dificultan aún más la integración fluida de esta tecnología en las redes actuales \cite{RubioGarcia:2023dgz} \cite{s11042-024-20535-x}.



\section{Implementación en microcontroladores Cortex-M4}
Finalmente, se revisan los resultados reportados en la literatura para implementaciones en dispositivos embebidos, con el objetivo de establecer un marco comparativo para las métricas obtenidas en el presente trabajo. La existencia de estos estudios previos valida, a su vez, la viabilidad técnica de desplegar algoritmos de criptografía postcuántica en plataformas de recursos restringidos.
\newline

Es relevante destacar que este análisis comparativo se centra específicamente en Kyber, Saber y \acrshort{hqc}. La elección de los dos primeros permite contrastar el estándar definitivo Kyber con su competidor más directo durante el proceso de selección Saber, ambos con arquitecturas de retículas muy similares. La decisión \acrshort{nist} se inclinó hacia Kyber principalmente por la mayor madurez teórica del problema \acrshort{lwe} frente al \acrshort{lwr} de Saber, y no por diferencias críticas de rendimiento. 
\newline

Por su parte, la inclusión de \acrshort{hqc} en este trabajo es fundamental para aportar diversidad criptográfica. Al ser el algoritmo basado en códigos seleccionado para la estandarización, representa la alternativa de seguridad más robusta en caso de que se descubran vulnerabilidades sistémicas en los modelos basados en retículas."


\subsection{Plataforma de Evaluación Común: El framework pqm4}
Para el análisis en dispositivos de recursos restringidos, se toma como referencia principal el proyecto \textbf{pqm4} \cite{pqm4_2019}, una plataforma de evaluación estandarizada que analiza el rendimiento de los algoritmos de la tercera ronda del \acrshort{nist}. Las pruebas de este framework se realizan sobre una placa STM32F4 Discovery, basada en un procesador ARM Cortex-M4 con 192 KB de RAM. La principal ventaja de utilizar esta fuente es la homogeneidad del entorno de pruebas: garantiza el uso de las mismas implementaciones para las primitivas comunes y el mismo generador de números aleatorios, asegurando así una comparación justa entre los distintos candidatos.
\newline

No obstante, es importante señalar que el banco de pruebas oficial de \textbf{pqm4} no incluye el algoritmo \acrshort{hqc}. Esto se debe a que sus requerimientos de memoria excedían la capacidad disponible (192 KB) de la placa STM32F4, haciendo inviable su implementación directa en dicho entorno. Por consiguiente, para incorporar \acrshort{hqc} a este análisis comparativo, se han extraído los métricas del estudio realizado en \cite{HQC_Optimized_2025}, donde se validó el algoritmo utilizando una placa de desarrollo NUCLEO-L4R5ZI, la cual dispone de una memoria superior de 640 KB de RAM, suficiente para albergar las claves y textos cifrados basados en códigos.
\newline

En la Tabla \ref{tab:algoritmos} se presentan los resultados provenientes de ambas fuentes para el nivel de seguridad de 256 bits que es el que se utiliza en este trabajo:

\begin{table}[H]
	\centering
	\begin{tabular}{llccc}
		\toprule
		\textbf{Algoritmo} & \textbf{Operación} & \textbf{Ciclos} & \textbf{Pila} (Bytes) & \textbf{Fuente} \\
		\midrule
		\multirow{3}{*}{Kyber-1024} 
		& KeyGen & 1,891,737 & 15,224 & \multirow{3}{*}{\cite{pqm4_2019}} \\
		& Encaps & 2,254,703 & 18,928 & \\
		& Decaps & 2,407,858 & 20,496 & \\
		\midrule
		\multirow{3}{*}{FireSaber} 
		& KeyGen & 3,815,672 & 20,144 & \multirow{3}{*}{\cite{pqm4_2019}} \\
		& Encaps & 4,745,405 & 23,008 & \\
		& Decaps & 5,402,295 & 24,592 & \\
		\midrule
		\multirow{3}{*}{HQC-256} 
		& KeyGen & 296,502,394 & 117,300 & \multirow{3}{*}{\cite{HQC_Optimized_2025}} \\
		& Encaps & 594,750,705 & 196,800 & \\
		& Decaps & 894,101,373 & 175,100 & \\
		\bottomrule
	\end{tabular}%
	\caption{Comparativa de rendimiento en un Cortex-M4 de Kyber-1024, FireSaber y HQC-256.}
	\label{tab:algoritmos}
\end{table}

Finalmente, es relevante destacar que los datos obtenidos provienen únicamente de las implementaciones de referencia y no de las versiones optimizadas para Cortex-M4. La razón de esta elección reside en que en este Trabajo de Fin de Grado se han utilizado las implementaciones limpias para garantizar su funcionamiento en cualquier dispositivo, evitando así la necesidad de adaptar el código ensamblador específico al microprocesador PSOC utilizado.

	
	\chapter{Fundamentos generales}
En este capítulo se desarrollan las bases matemáticas de los distintos algoritmos a implementar.
\section{Introducción}
 
	
	\chapter{Desarrollo}
\section{Implementación comunicación serie}
\subsection{Parámetros generales y formato mensajes}
\subsection{Implementación en el ordenador}
\subsection{Implementación en el microprocesador}
\section{Estructura y API para los algoritmos de cifrado asimétrico}
\subsection{Kyber}
\subsection{Saber}
\subsection{Bike}
\subsection{HQC}
\section{Implementación algoritmos en el PC}
\subsection{Compilación en librerías}
\subsection{Diagrama de uso}
\subsection{Diagrama funcional}
\subsection{Diagrama de clases}
\section{Implementación de algoritmos en el microcontrolador}
\subsection{Diagrama de uso}
\subsection{Diagrama funcional}

\section{Implementación del intercambio de claves. Creación del secreto compartido}
Hablar de los modelos de comunicaciones a implementar .... (msg teams)
\subsection{Modelo 1}
\subsection{Modelo 2}
\subsection{Modelo 3}
\section{Tests de rendimiento realizados}
	
	\chapter{Resultados y discusión}

En este capítulo se muestran los resultados obtenidos de aplicar las rutinas desarrolladas con anterioridad.


\section{Resultados}
\subsection{(No sé si lo metere) Resultado de ejecución de los algoritmos clásicos}
\subsubsection{\gls{rsa}}
\subsubsection{\gls{ecc}}
\subsection{Resultados de ejecución de los algoritmos post-cuánticos}
\subsubsection{Kyber}
\subsubsection{Saber}
\subsubsection{HQC}
\subsubsection{Bike}
\subsection{Resultados de los distintos modelos de comunicación}
\subsubsection{Modelo 1}
\subsubsection{Modelo 2}
\subsubsection{Modelo 3}
\section{Discusión}
\subsection{Comparativa entre los distintos algoritmos post-cuánticos analizados}
\subsection{Comparativa entre los distintos modelos de comunicación}
	
	\chapter{Conclusiones}

Se presentan a continuación las conclusiones del proyecto y desarrollos futuros para mejorar la implementación.

\section{Conclusión}

Una vez finalizado el proyecto...

\section{Desarrollos futuros}

Un posible desarrollo...
	
	
	
	%Glosario, lista de símbolos, notas, etc.
	\appendix
	
	\chapter{Definiciones básicas}
\label{chap:def}
\chapter{Ejemplo \gls{rlwe}}
\label{chap:ej:rlwe}
Sea el espacio de trabajo en $R_{17}=\mathbb{Z}_{17}[X]/\left(X^2+1\right)$ con un mensaje $z\in \{0,1\}^2$ y la distribución del error $e\in \{-1,0,1\}$.
\newline

En el primer paso se generan $a, s$ y $e$:
\begin{equation}
	\begin{array}{l}
		a[X]=3+4X\\
		s[X]=1+0X\\
		e[X]=-1+1X
	\end{array}
\end{equation}

Una vez inicializados los parámetros, se procede al cálculo de \(b\). La reducción módulo $X^2+1$ equivale a sustituir $X^2$ por $-1$ cada vez que aparezca.
\begin{equation}
	b[X]=a[X]\cdot s[X]+e[X]=2+5X
\end{equation}

Con la clave pública calculada \(a||b\) se puede cifrar un mensaje \(z\), pero antes se generan los valores de \(z, r, e_1\) y \(e_2\):
\begin{equation}
	\begin{array}{l}
		z[X]=1+0X\\
		r[X]=1+1X\\
		e_1[X]=0+1X\\
		e_2[X]=-1+0X
	\end{array}
\end{equation}

Con estos valores se calculan los textos cifrados:
\begin{equation}
	\begin{array}{l}
		u[X]=a[X]\cdot r[X]+e_1[X]=16+8X\\
		v[X]=b[X]\cdot r[X]+e_2[X]+\left\lfloor \dfrac{q}{2} \right\rceil \cdot z[X] =5+7X
	\end{array}
\end{equation}

Por último, se comprueba que el mensaje se descifra correctamente. 
\begin{equation}
	z'[X]=v[X]-u[X]\cdot s[X]=6+16X \rightarrow \left\{ \begin{array}{l}
		z'_0: \ \begin{array}{l}
			\text{d}_0(0) = 6 \\
			\text{d}_0(9) = 3
		\end{array} \\
		z'_1: \ \begin{array}{l}
			\text{d}_1(0) = 1 \\
			\text{d}_1(9) = 8
		\end{array}
	\end{array} \right.
\end{equation}

Con estas distancia se obtiene que \(z=(1,0)\). No obstante, aunque este descifrado se comprueba que se cumple que el error no supera la magnitud límite $q/4=4.25$.
\begin{equation}
	\varepsilon [X]=r[X]\cdot e[X]- s[X]\cdot e_1[X] +e_2[X]=14+16X
\end{equation}

Para cumplirse la distancia a \(0\) debe ser menor a $q/4$ para cada coeficiente:
\begin{equation}
	\begin{array}{l}
		\text{d}_0(0) = 3 \\
		\text{d}_1(0) = 1
	\end{array} 
\end{equation}
	
	\backmatter
	%estilo de bibliografía: plana, alfa...
	\bibliographystyle{ieeetr}
	
	%genera doble hoja en blanco
	\cleardoublepage
	
	
	%apartado de bibliografía
	\addcontentsline{toc}{chapter}{Bibliografia}
	
	%se incluye la bibliografía. Archivo de tipo .bib (bibtex)
	\bibliography{bibliografia/bibliografia}
	
	%fin del documento
\end{document}